\documentclass[serif, 9pt, aspectratio=32]{beamer} 
\usetheme{Darmstadt}

\usepackage{appendixnumberbeamer}
\usepackage{booktabs}
\usepackage[scale=2]{ccicons}
\usepackage{pgfplots}
\usepgfplotslibrary{dateplot}
\usepackage{xspace}
\usepackage{tikz}
\usepackage{hyperref}
\usepackage{xcolor}
\usepackage{listings}
\usetikzlibrary{shapes, arrows}
\newcommand{\themename}{\textbf{\textsc{metropolis}}\xspace}

\title{Introduction to Coding}
\date{\today}
\author{Tan Sein Jone}
\institute{University of British Columbia}

\pgfplotsset{compat=1.18}
\setbeamertemplate{footline}[frame number]

\begin{document}

\maketitle

\begin{frame}{Table of contents}
    \setbeamertemplate{section in toc}[sections numbered]
    \tableofcontents[hideallsubsections]
\end{frame}

\section{Functions}

\begin{frame}
    \frametitle{Table of Contents}
    \setbeamertemplate{section in toc}[sections numbered]
    \tableofcontents[currentsection]
\end{frame}

\begin{frame}
    \centering
    \frametitle{Hello World}
    \begin{itemize}
        \setlength{\itemsep}{3em}
        \item The first program that every programmer writes
        \item How to start a Python program
        \item How to print to the console
    \end{itemize}
\end{frame}

\begin{frame}[fragile]
    \frametitle{In Terminal}
    \begin{lstlisting}[language=Python]
        touch hello_world.py
    \end{lstlisting}
\end{frame}

\begin{frame}[fragile]
    \frametitle{hello\_world.py}
    \begin{lstlisting}[language=Python]
        print("Hello, World!")
    \end{lstlisting}
\end{frame}

\section{Variables}

\begin{frame}
    \frametitle{Table of Contents}
    \setbeamertemplate{section in toc}[sections numbered]
    \tableofcontents[currentsection]
\end{frame}

\begin{frame}
    \centering
    \frametitle{What are Variables?}
    \begin{itemize}
        \setlength{\itemsep}{3em}
        \item Variables are used to store data
        \item Variables are assigned a value
        \item Variables can be changed
    \end{itemize}
\end{frame}

\begin{frame}
    \centering
    \frametitle{Variable Naming Rules}
    \begin{itemize}
        \setlength{\itemsep}{3em}
        \item Variables must start with a letter or underscore
        \item Variables can only contain letters, numbers, and underscores
        \item Variables are case-sensitive
        \item Variables cannot be reserved words
    \end{itemize}
\end{frame}

\begin{frame}
    \centering
    \frametitle{Variable Naming Conventions}
    \begin{itemize}
        \setlength{\itemsep}{3em}
        \item Camel Case: myVariableName
        \item Pascal Case: MyVariableName
        \item Snake Case: my\_variable\_name
    \end{itemize}
\end{frame}

\begin{frame}
    \centering
    \frametitle{Scope}
    \begin{itemize}
        \setlength{\itemsep}{3em}
        \item Global Variables: Variables declared outside of a function
              \begin{itemize}
                  \item Can be accessed anywhere
              \end{itemize}
        \item Local Variables: Variables declared inside of a function
              \begin{itemize}
                  \item Can only be accessed within the function
              \end{itemize}
    \end{itemize}
\end{frame}

\begin{frame}[fragile]
    \begin{lstlisting}[language=Python]
        def my_function()\:
            x = 10
        x = 20
        my_function()
        print(x)
    \end{lstlisting}
\end{frame}

\section{Data Types}

\begin{frame}
    \frametitle{Table of Contents}
    \setbeamertemplate{section in toc}[sections numbered]
    \tableofcontents[currentsection]
\end{frame}

\begin{frame}
    \centering
    \frametitle{Data Types}
    \begin{itemize}
        \setlength{\itemsep}{1em}
        \item Integers: Whole numbers
        \item Floats: Numbers with decimals
        \item Strings: Text
        \item Booleans: True or False
        \item Lists: Ordered collection of items
        \item Tuples: Ordered collection of items that cannot be changed
        \item Dictionaries: Unordered collection of items
        \item Sets: Unordered collection of unique items
    \end{itemize}
\end{frame}

\begin{frame}
    \centering
    \frametitle{Importance of Type Checking}
    \begin{itemize}
        \setlength{\itemsep}{3em}
        \item There are advantages and disadvantages to using each data type
        \item Interacting with different data types can cause errors
    \end{itemize}
\end{frame}

\begin{frame}[fragile]
    \begin{lstlisting}[language=Python]
        random_int = 10
        random_float = 10.0
        random_string = "10"
    \end{lstlisting}
\end{frame}

\begin{frame}
    \centering
    \frametitle{Type Casting}
    \begin{itemize}
        \setlength{\itemsep}{3em}
        \item Converting between data types
        \item Can be done using built-in functions
        \item Not all conversions are possible
    \end{itemize}
\end{frame}

\begin{frame}[fragile]
    \begin{lstlisting}[language=Python]
        converted_int = int(random_float)
    \end{lstlisting}
\end{frame}

\begin{frame}
    \centering
    \frametitle{Data Types with Multiple Values}
    \begin{itemize}
        \setlength{\itemsep}{3em}
        \item Lists, Tuples, Dictionaries, and Sets can store multiple values
        \item Each value can be a different data type
        \item Each value can be accessed using an index
        \item Each value can be changed
    \end{itemize}
\end{frame}

\begin{frame}[fragile]
    \begin{lstlisting}[language=Python]
        random_list = [10, 10.0, "10"]
        random_tuple = (10, 10.0, "10")
        random_dict = {"int": 10, "float": 10.0, "string": "10"}
        random_set = {10, 10.0, "10"}
    \end{lstlisting}
\end{frame}

\begin{frame}
    \centering
    \frametitle{Common Usecases}
    \begin{itemize}
        \setlength{\itemsep}{3em}
        \item Lists and dicitonaries will likely be the most used data types
        \item Lists are used to store multiple values
        \item Dictionaries are used to store key-value pairs
    \end{itemize}
\end{frame}

\begin{frame}
    \centering
    \frametitle{Combining Data Types}
    \begin{itemize}
        \setlength{\itemsep}{3em}
        \item Data types can be combined
        \item Lists can store dictionaries
        \item Dictionaries can store lists
    \end{itemize}
\end{frame}

\begin{frame}
    \centering
    \frametitle{Lists of dictionaries}
    \begin{itemize}
        \setlength{\itemsep}{3em}
        \item Say for example we have a list of students
        \item Each student has a name, age, and grade
        \item We can store this information in a list of dictionaries
        \item Each dictionary will represent a student
    \end{itemize}
\end{frame}

\begin{frame}[fragile]
    \begin{lstlisting}
        students = [
            {"name": "Alice", "age": 20, "grade": 90},
            {"name": "Bob", "age": 21, "grade": 85},
            {"name": "Charlie", "age": 22, "grade": 80}
        ]
    \end{lstlisting}
\end{frame}

\begin{frame}
    \centering
    \frametitle{Dictionaries of lists}
    \begin{itemize}
        \setlength{\itemsep}{3em}
        \item Say for example we have a dictionary of students
        \item Each student has a list of grades
        \item We can store this information in a dictionary of lists
        \item Each key will represent a student
    \end{itemize}
\end{frame}

\begin{frame}[fragile]
    \begin{lstlisting}
        students = {
            "Alice": [90, 85, 80],
            "Bob": [85, 80, 75],
            "Charlie": [80, 75, 70]
        }
    \end{lstlisting}
\end{frame}

\begin{frame}
    \centering
    \frametitle{Common methods for lists and dictionaries}
    \begin{itemize}
        \setlength{\itemsep}{2em}
        \item append(): Adds an element to the end of the list
        \item insert(): Adds an element at a specific index
        \item remove(): Removes an element from the list
        \item get(): Gets the value of a key in a dictionary
        \item keys(): Gets all the keys in a dictionary
        \item values(): Gets all the values in a dictionary
    \end{itemize}
\end{frame}

\section{Conditionals}

\begin{frame}
    \frametitle{Table of Contents}
    \setbeamertemplate{section in toc}[sections numbered]
    \tableofcontents[currentsection]
\end{frame}

\begin{frame}
    \centering
    \frametitle{What are Conditionals?}
    \begin{itemize}
        \setlength{\itemsep}{3em}
        \item Conditionals are used to make decisions
        \item Conditionals are used to execute code based on a condition
        \item Conditionals are used to compare values
    \end{itemize}
\end{frame}

\begin{frame}
    \centering
    \frametitle{Comparison Operators}
    \begin{itemize}
        \setlength{\itemsep}{3em}
        \item ==: Equal to
        \item !=: Not equal to
        \item \textless: Less than
        \item \textgreater: Greater than
        \item \textless=: Less than or equal to
        \item \textgreater=: Greater than or equal to
    \end{itemize}
\end{frame}

\begin{frame}
    \centering
    \frametitle{Logical Operators}
    \begin{itemize}
        \setlength{\itemsep}{3em}
        \item and: Returns True if both statements are true
        \item or: Returns True if one of the statements is true
        \item not: Returns True if the statement is false
    \end{itemize}
\end{frame}

\begin{frame}
    \centering
    \frametitle{If Statements}
    \begin{itemize}
        \setlength{\itemsep}{3em}
        \item If statements are used to execute code if a condition is true
        \item If statements can be followed by an else statement
        \item If statements can be followed by an elif statement
    \end{itemize}
\end{frame}

\begin{frame}[fragile]
    \begin{lstlisting}[language=Python]
        x = 10
        if x == 10:
            print("x is 10")
        elif x == 20:
            print("x is 20")
        else:
            print("x is not 10 or 20")
    \end{lstlisting}
\end{frame}

\section{Loops}

\begin{frame}
    \frametitle{Table of Contents}
    \setbeamertemplate{section in toc}[sections numbered]
    \tableofcontents[currentsection]
\end{frame}

\begin{frame}
    \centering
    \frametitle{What are Loops?}
    \begin{itemize}
        \setlength{\itemsep}{3em}
        \item Loops are used to repeat code
        \item Loops are used to iterate over a sequence
        \item Loops are used to execute code a specific number of times
    \end{itemize}
\end{frame}

\begin{frame}
    \centering
    \frametitle{For Loops}
    \begin{itemize}
        \setlength{\itemsep}{3em}
        \item For loops are used to iterate over a sequence
        \item For loops are used to execute code a specific number of times
        \item For loops can be used with lists, tuples, dictionaries, and sets
    \end{itemize}
\end{frame}

\begin{frame}[fragile]
    \begin{lstlisting}[language=Python]
        for x in range(10):
            print(x)
    \end{lstlisting}
\end{frame}

\begin{frame}
    \centering
    \frametitle{While Loops}
    \begin{itemize}
        \setlength{\itemsep}{3em}
        \item While loops are used to execute code as long as a condition is true
        \item While loops are used to execute code a specific number of times
        \item While loops can be used with lists, tuples, dictionaries, and sets
    \end{itemize}
\end{frame}

\begin{frame}[fragile]
    \begin{lstlisting}[language=Python]
        x = 0
        while x < 10:
            print(x)
            x += 1
    \end{lstlisting}
\end{frame}

\end{document}