\documentclass[serif, 9pt, aspectratio=32]{beamer} 
\usetheme{Darmstadt}

\usepackage{appendixnumberbeamer}
\usepackage{booktabs}
\usepackage[scale=2]{ccicons}
\usepackage{pgfplots}
\usepgfplotslibrary{dateplot}
\usepackage{xspace}
\usepackage{tikz}
\usepackage{hyperref}
\usepackage{xcolor}
\usepackage{listings}
\usetikzlibrary{shapes, arrows}
\newcommand{\themename}{\textbf{\textsc{metropolis}}\xspace}

\title{Introduction to Coding}
\date{\today}
\author{Tan Sein Jone}
\institute{University of British Columbia}

\pgfplotsset{compat=1.18}
\setbeamertemplate{footline}[frame number]

\begin{document}

\maketitle

\begin{frame}{Table of contents}
    \setbeamertemplate{section in toc}[sections numbered]
    \tableofcontents[hideallsubsections]
\end{frame}

\section{How to Write a Function}

\begin{frame}
    \frametitle{Table of Contents}
    \setbeamertemplate{section in toc}[sections numbered]
    \tableofcontents[currentsection]
\end{frame}

\begin{frame}
    \centering
    \frametitle{What are Functions?}
    \begin{itemize}
        \setlength{\itemsep}{3em}
        \item Functions are used to organize code
        \item Functions are used to make code reusable
        \item Functions are used to make code easier to read
    \end{itemize}
\end{frame}

\begin{frame}
    \centering
    \frametitle{Defining Functions}
    \begin{itemize}
        \setlength{\itemsep}{3em}
        \item Functions are defined using the def keyword
        \item Functions can take arguments
        \item Functions can return values
    \end{itemize}
\end{frame}

\begin{frame}[fragile]
    \begin{lstlisting}[language=Python]
        def my_function(x):
            return x
    \end{lstlisting}
\end{frame}

\begin{frame}
    \centering
    \frametitle{Calling Functions}
    \begin{itemize}
        \setlength{\itemsep}{3em}
        \item Functions are called using the function name
        \item Functions can be called with or without arguments
        \item Functions can be called multiple times
    \end{itemize}
\end{frame}

\begin{frame}[fragile]
    \begin{lstlisting}[language=Python]
        def my_function(x):
            return x
        print(my_function(10))
    \end{lstlisting}
\end{frame}

\begin{frame}
    \centering
    \frametitle{Pseudocode}
    \begin{itemize}
        \setlength{\itemsep}{3em}
        \item Pseudocode is used to plan out code
        \item Pseudocode is used to break down complex problems
        \item Pseudocode is used to make code easier to write
    \end{itemize}
\end{frame}

\begin{frame}[fragile]
    \begin{lstlisting}[language=Python]
        # Pseudocode
        # Define a function that takes a list of numbers as 
            an argument
        # Iterate over the list of numbers
        # If the number is even, add it to a new list
        # Return the new list
    \end{lstlisting}
\end{frame}

\begin{frame}[fragile]
    \begin{lstlisting}[language=Python]
        def even_numbers(numbers):
            even_numbers = []
            for number in numbers:
                if number % 2 == 0:
                    even_numbers.append(number)
            return even_numbers
    \end{lstlisting}
\end{frame}

\begin{frame}
    \centering
    \frametitle{Debugging}
    \begin{itemize}
        \setlength{\itemsep}{3em}
        \item Debugging is the process of finding and fixing errors in code
        \item Debugging is an important skill for programmers
        \item It may be frustrating to get an error message, but sometimes not getting one can be worse
        \item When we get an error message, we can use it to help us find the problem
        \item When the program runs without errors, but the output is not what we expect, have to use debugging techniques to find the problem
    \end{itemize}
\end{frame}

\begin{frame}
    \centering
    \frametitle{Read Error Messages}
    \begin{itemize}
        \setlength{\itemsep}{3em}
        \item Syntax Errors: Errors in the code structure
        \item Logic Errors: Errors in the code logic
        \item Runtime Errors: Errors that occur while the code is running
    \end{itemize}
\end{frame}

\begin{frame}
    \centering
    \frametitle{Rubber Duck Debugging}
    \begin{itemize}
        \setlength{\itemsep}{3em}
        \item Explaining the code to someone else
        \item Explaining the code to an inanimate object
        \item Explaining the code to yourself
    \end{itemize}
\end{frame}

\section{Simple Functions}

\begin{frame}
    \frametitle{Table of Contents}
    \setbeamertemplate{section in toc}[sections numbered]
    \tableofcontents[currentsection]
\end{frame}

\begin{frame}
    \centering
    \frametitle{Simple Functions}
    \begin{itemize}
        \setlength{\itemsep}{3em}
        \item Let's write a simple function that does only one thing
        \item This function will take a number as an argument and return the square of that number
    \end{itemize}
\end{frame}

\begin{frame}[fragile]
    \begin{lstlisting}[language=Python]
        def square(x):
            return x * x
    \end{lstlisting}
\end{frame}

\begin{frame}
    \centering
    \frametitle{Simple Functions}
    \begin{itemize}
        \setlength{\itemsep}{3em}
        \item Functions don't necessarily have to take arguments
        \item They don't necessarily have to return values
    \end{itemize}
\end{frame}

\begin{frame}[fragile]
    \begin{lstlisting}[language=Python]
        def hello():
            print("Hello, World!")
    \end{lstlisting}
\end{frame}

\begin{frame}
    \centering
    \frametitle{When Will We Use Simple Functions?}
    \begin{itemize}
        \setlength{\itemsep}{3em}
        \item When we want to break down a complex problem into smaller parts
        \item When we want to make our code more readable
    \end{itemize}
\end{frame}

\section{Compound Functions}

\begin{frame}
    \frametitle{Table of Contents}
    \setbeamertemplate{section in toc}[sections numbered]
    \tableofcontents[currentsection]
\end{frame}

\begin{frame}
    \centering
    \frametitle{Compound Functions}
    \begin{itemize}
        \setlength{\itemsep}{3em}
        \item You can write functions that call other functions
        \item This is called a compound function
        \item Say you want to write a function that first checks if a value is an integer, and then squares it
    \end{itemize}
\end{frame}

\begin{frame}[fragile]
    \begin{lstlisting}[language=Python]
        def is_integer(x):
            return type(x) == int

        def square(x):
            return x * x

        def square_integer(x):
            if is_integer(x):
                return square(x)
            else:
                return "Not an integer"
    \end{lstlisting}
\end{frame}

\begin{frame}
    \centering
    \frametitle{Importance of Modularity}
    \begin{itemize}
        \setlength{\itemsep}{3em}
        \item Modularity is the practice of breaking down code into smaller, more manageable parts
        \item Modularity makes code easier to read and understand
        \item Modularity makes code easier to maintain
        \item Modularity makes code easier to test
    \end{itemize}
\end{frame}

\end{document}