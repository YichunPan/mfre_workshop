\documentclass[serif, 9pt, aspectratio=32]{beamer} 
\usetheme{Darmstadt}

\usepackage{appendixnumberbeamer}
\usepackage{booktabs}
\usepackage[scale=2]{ccicons}
\usepackage{pgfplots}
\usepgfplotslibrary{dateplot}
\usepackage{xspace}
\usepackage{tikz}
\usepackage{hyperref}
\usepackage{xcolor}
\usepackage{listings}
\usetikzlibrary{shapes, arrows}
\newcommand{\themename}{\textbf{\textsc{metropolis}}\xspace}

\title{Introduction to Python}
\date{\today}
\author{Tan Sein Jone}
\institute{University of British Columbia}

\pgfplotsset{compat=1.18}
\setbeamertemplate{footline}[frame number]

\begin{document}

\maketitle

\begin{frame}{Table of contents}
    \setbeamertemplate{section in toc}[sections numbered]
    \tableofcontents[hideallsubsections]
\end{frame}

\section{Statistical Analysis}

\begin{frame}
    \frametitle{Table of Contents}
    \setbeamertemplate{section in toc}[sections numbered]
    \tableofcontents[currentsection]
\end{frame}

\begin{frame}
    \centering
    \frametitle{Statasmodel}
    \begin{itemize}
        \setlength{\itemsep}{2em}
        \item Statsmodels is a library for statistical modeling
        \item It provides support for linear regression, logistic regression, and time series analysis
        \item It allows you to fit models, make predictions, and evaluate results
    \end{itemize}
\end{frame}

\begin{frame}
    \centering
    \frametitle{Linear Regression}
    \begin{itemize}
        \setlength{\itemsep}{2em}
        \item Linear regression is a statistical method for modeling the relationship between two variables
        \item It is used to predict the value of one variable based on the value of another variable
        \item It is used to estimate the coefficients of the regression equation
    \end{itemize}
\end{frame}

\begin{frame}[fragile]
    \frametitle{Linear Regression}
    \begin{lstlisting}[language=Python]
import statsmodels.api as sm
import numpy as np
import random

data = {
    "age": np.random.normal(50, 10, 100),
    "income": np.random.normal(50000, 10000, 100)
}
data = pd.DataFrame(data)
X = data["age"]
y = data["income"]
X = sm.add_constant(X)
model = sm.OLS(y, X).fit()
print(model.summary())
    \end{lstlisting}
\end{frame}

\begin{frame}
    \centering
    \frametitle{Hypothesis Testing}
    \begin{itemize}
        \setlength{\itemsep}{2em}
        \item Hypothesis testing is a statistical method for testing the validity of a hypothesis
        \item It is used to determine whether a hypothesis is true or false
        \item It is used to make inferences about a population based on a sample
    \end{itemize}
\end{frame}

\begin{frame}[fragile]
    \frametitle{Hypothesis Testing}
    \begin{lstlisting}[language=Python]
import statsmodels.api as sm

data = {
    "age": np.random.normal(50, 10, 100),
    "income": np.random.normal(50000, 10000, 100)
}

data = pd.DataFrame(data)
X = data["age"]
y = data["income"]
X = sm.add_constant(X)
model = sm.OLS(y, X).fit()
print(model.summary())
print(model.pvalues)
    \end{lstlisting}
\end{frame}

\section{Data Visualization}

\begin{frame}
    \frametitle{Table of Contents}
    \setbeamertemplate{section in toc}[sections numbered]
    \tableofcontents[currentsection]
\end{frame}

\begin{frame}
    \centering
    \frametitle{Matplotlib}
    \begin{itemize}
        \setlength{\itemsep}{2em}
        \item Matplotlib is a library for creating static, animated, and interactive visualizations
        \item It provides support for line plots, bar plots, scatter plots, and histograms
        \item It allows you to customize the appearance of plots
    \end{itemize}
\end{frame}

\begin{frame}
    \centering
    \frametitle{Line Plot}
    \begin{itemize}
        \setlength{\itemsep}{2em}
        \item A line plot is a type of plot that displays data as a series of points connected by lines
        \item It is used to show trends, patterns, and relationships in data
        \item It is used to visualize the relationship between two variables
    \end{itemize}
\end{frame}

\begin{frame}[fragile]
    \frametitle{Line Plot}
    \begin{lstlisting}[language=Python]
import matplotlib.pyplot as plt
import numpy as np

x = np.linspace(0, 10, 100)
y = np.sin(x)
plt.plot(x, y)
plt.show()
    \end{lstlisting}
\end{frame}

\begin{frame}
    \centering
    \frametitle{Seaborn}
    \begin{itemize}
        \setlength{\itemsep}{2em}
        \item Seaborn is a library for creating statistical data visualizations
        \item It provides support for line plots, bar plots, scatter plots, and histograms
        \item It allows you to customize the appearance of plots
    \end{itemize}
\end{frame}

\begin{frame}
    \centering
    \frametitle{Bar Plot}
    \begin{itemize}
        \setlength{\itemsep}{2em}
        \item A bar plot is a type of plot that displays data as a series of bars
        \item It is used to compare the values of different categories
        \item It is used to visualize the distribution of a categorical variable
    \end{itemize}
\end{frame}

\begin{frame}[fragile]
    \frametitle{Bar Plot}
    \begin{lstlisting}[language=Python]
import seaborn as sns

data = {
    "age": np.random.normal(50, 10, 100),
    "income": np.random.normal(50000, 10000, 100)
}
data = pd.DataFrame(data)
sns.barplot(x="age", y="income", data=data)
plt.savefig("barplot.png")
plt.show()
    \end{lstlisting}
\end{frame}

\section{Pandas Best Practices}

\begin{frame}
    \frametitle{Table of Contents}
    \setbeamertemplate{section in toc}[sections numbered]
    \tableofcontents[currentsection]
\end{frame}

\begin{frame}
    \centering
    \frametitle{Pandas Best Practices}
    \begin{itemize}
        \setlength{\itemsep}{2em}
        \item Use the \texttt{read\_csv} function to read CSV files
        \item Use the \texttt{head} function to display the first few rows of a DataFrame
        \item Use the \texttt{info} function to display information about a DataFrame
        \item Use the \texttt{describe} function to display summary statistics of a DataFrame
    \end{itemize}
\end{frame}

\begin{frame}
    \centering
    \frametitle{Pandas Best Practices}
    \begin{itemize}
        \setlength{\itemsep}{2em}
        \item Always have an original dataframe for your import data
        \item Use the \texttt{copy} function to create a copy of a DataFrame
        \item Write to csv only when necessary
        \item You may run into memory issues if you have a large dataset
    \end{itemize}
\end{frame}

\begin{frame}
    \centering
    \frametitle{Pandas Best Practices}
    \begin{itemize}
        \setlength{\itemsep}{2em}
        \item It's always a good idea to save memory by overriding unused variables
        \item Restart the kernel if you run into memory issues
        \item Use the \texttt{apply} function when you need to apply a function to a DataFrame
    \end{itemize}
\end{frame}

\begin{frame}[fragile]
    \frametitle{Pandas Best Practices}
    \begin{lstlisting}[language=Python]
import pandas as pd

df = pd.read_csv("../data/province_weather.csv")
df.head()
df.info()

weather = df.copy()
weather.describe()
    \end{lstlisting}
\end{frame}

\begin{frame}[fragile]
    \frametitle{Pandas Best Practices}
    \begin{lstlisting}[language=Python]
weather['fahrenheit'] = weather['Temperature']
                            .apply(lambda x: x * 9/5 + 32)

weather.to_csv(
    "../data/province_weather_fahrenheit.csv", index=False
)
    \end{lstlisting}
\end{frame}

\end{document}