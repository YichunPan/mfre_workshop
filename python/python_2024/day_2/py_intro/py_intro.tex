\documentclass[serif, 9pt, aspectratio=32]{beamer} 
\usetheme{Darmstadt}

\usepackage{appendixnumberbeamer}
\usepackage{booktabs}
\usepackage[scale=2]{ccicons}
\usepackage{pgfplots}
\usepgfplotslibrary{dateplot}
\usepackage{xspace}
\usepackage{tikz}
\usepackage{hyperref}
\usepackage{xcolor}
\usepackage{listings}
\usetikzlibrary{shapes, arrows}
\newcommand{\themename}{\textbf{\textsc{metropolis}}\xspace}

\title{Introduction to Python}
\date{\today}
\author{Tan Sein Jone}
\institute{University of British Columbia}

\pgfplotsset{compat=1.18}
\setbeamertemplate{footline}[frame number]

\begin{document}

\maketitle

\begin{frame}{Table of contents}
    \setbeamertemplate{section in toc}[sections numbered]
    \tableofcontents[hideallsubsections]
\end{frame}

\section{Object Oriented Programming}

\begin{frame}
    \frametitle{Table of Contents}
    \setbeamertemplate{section in toc}[sections numbered]
    \tableofcontents[currentsection]
\end{frame}

\begin{frame}
    \centering
    \frametitle{Classes and Objects}
    \begin{itemize}
        \setlength{\itemsep}{2em}
        \item A class is a blueprint for creating object
        \item Classes define the properties and behaviours of objects
        \item Objects have attributes and methods
        \item Attributes are variables that store data
        \item Methods are functions that perform actions
    \end{itemize}
\end{frame}

\begin{frame}
    \centering
    \frametitle{Attributes}
    \begin{itemize}
        \setlength{\itemsep}{2em}
        \item Attributes are variables that store data
        \item They are defined in the \texttt{\_\_init\_\_} method
        \item They are accessed using the dot operator
    \end{itemize}
\end{frame}

\begin{frame}[fragile]
    \frametitle{Attributes}
    \begin{lstlisting}[language=Python]
class Person:
    def __init__(self, name, age):
        self.name = name
        self.age = age
    \end{lstlisting}
\end{frame}

\begin{frame}[fragile]
    \frametitle{Attributes}
    \begin{lstlisting}[language=Python]
person = Person("John", 36)
print(person.name)
print(person.age)
    \end{lstlisting}
\end{frame}

\begin{frame}
    \centering
    \frametitle{Methods}
    \begin{itemize}
        \setlength{\itemsep}{2em}
        \item Methods are functions that perform actions
        \item They are defined in the class
        \item They are accessed using the dot operator
    \end{itemize}
\end{frame}

\begin{frame}[fragile]
    \frametitle{Methods}
    \begin{lstlisting}[language=Python]
class Person:
    def __init__(self, name, age):
        self.name = name
        self.age = age

    def greet(self):
        print(f"Hello, my name is {self.name}")
    \end{lstlisting}
\end{frame}

\begin{frame}[fragile]
    \frametitle{Methods}
    \begin{lstlisting}[language=Python]
person = Person("John", 36)
person.greet()
    \end{lstlisting}
\end{frame}

\begin{frame}
    \centering
    \frametitle{Benefits of OOP}
    \begin{itemize}
        \setlength{\itemsep}{2em}
        \item Encapsulation
        \item Inheritance
        \item Polymorphism
    \end{itemize}
\end{frame}

\begin{frame}
    \centering
    \frametitle{Encapsulation}
    \begin{itemize}
        \setlength{\itemsep}{2em}
        \item Encapsulation is the bundling of data and methods that operate on the data
        \item It restricts access to some of the object's components
        \item It prevents the accidental modification of data
    \end{itemize}
\end{frame}

\begin{frame}
    \centering
    \frametitle{Inheritance}
    \begin{itemize}
        \setlength{\itemsep}{2em}
        \item Inheritance is the mechanism of basing a class upon another class
        \item It allows a class to inherit attributes and methods from another class
        \item It allows a class to override methods of another class
    \end{itemize}
\end{frame}

\begin{frame}
    \centering
    \frametitle{Polymorphism}
    \begin{itemize}
        \setlength{\itemsep}{2em}
        \item Polymorphism is the ability to present the same interface for different data types
        \item It allows a function to accept different data types
        \item It allows a class to override methods of another class
    \end{itemize}
\end{frame}

\section{Pythonics}

\begin{frame}
    \frametitle{Table of Contents}
    \setbeamertemplate{section in toc}[sections numbered]
    \tableofcontents[currentsection]
\end{frame}

\begin{frame}
    \centering
    \frametitle{Pythonics}
    \begin{itemize}
        \setlength{\itemsep}{2em}
        \item Pythonic code is code that follows the conventions of the Python language
        \item It is code that is clean, readable, and maintainable
        \item It is code that is idiomatic and expressive
    \end{itemize}
\end{frame}

\begin{frame}
    \centering
    \frametitle{Zen of Python}
    \begin{itemize}
        \setlength{\itemsep}{2em}
        \item The Zen of Python is a collection of aphorisms that capture the philosophy of Python
        \item It is a set of guiding principles for writing computer programs
        \item It is a set of rules for writing Pythonic code
    \end{itemize}
\end{frame}

\begin{frame}
    \centering
    \frametitle{Value Swapping and Multiple Assignment}
    \begin{itemize}
        \setlength{\itemsep}{2em}
        \item Python allows you to swap the values of two variables in a single line
        \item It also allows you to assign multiple values to multiple variables in a single line
    \end{itemize}
\end{frame}

\begin{frame}[fragile]
    \frametitle{Value Swapping and Multiple Assignment}
    \begin{lstlisting}[language=Python]
a = 1
b = 2
a, b = b, a
print(a, b)
    \end{lstlisting}
\end{frame}

\begin{frame}
    \centering
    \frametitle{Passing Multiple Arguments}
    \begin{itemize}
        \setlength{\itemsep}{2em}
        \item Python allows you to pass multiple arguments to a function
        \item It also allows you to pass keyword arguments to a function
    \end{itemize}
\end{frame}

\begin{frame}[fragile]
    \frametitle{Passing Multiple Arguments}
    \begin{lstlisting}[language=Python]
def greet(*names):
    for name in names:
        print(f"Hello, {name}")

greet("John", "Jane", "Jack")
    \end{lstlisting}
\end{frame}

\begin{frame}
    \centering
    \frametitle{List Comprehension}
    \begin{itemize}
        \setlength{\itemsep}{2em}
        \item List comprehension is a concise way to create lists
        \item It allows you to create lists using a single line of code
        \item It is more readable and expressive than traditional loops
    \end{itemize}
\end{frame}

\begin{frame}[fragile]
    \frametitle{List Comprehension}
    \begin{lstlisting}[language=Python]
squares = [x ** 2 for x in range(10)]
print(squares)
    \end{lstlisting}
\end{frame}

\begin{frame}
    \centering
    \frametitle{A Note on Indentation}
    \begin{itemize}
        \setlength{\itemsep}{2em}
        \item Python uses indentation to define blocks of code
        \item It uses whitespace to delimit code
    \end{itemize}
\end{frame}

\section{Data Handling}

\begin{frame}
    \frametitle{Table of Contents}
    \setbeamertemplate{section in toc}[sections numbered]
    \tableofcontents[currentsection]
\end{frame}

\begin{frame}
    \centering
    \frametitle{Data Handling}
    \begin{itemize}
        \setlength{\itemsep}{2em}
        \item Data handling is the process of managing data
        \item It involves reading, writing, and processing data
        \item It involves working with files, databases, and APIs
    \end{itemize}
\end{frame}

\begin{frame}
    \centering
    \frametitle{Lists and Dictionaries}
    \begin{itemize}
        \setlength{\itemsep}{2em}
        \item Lists are ordered collections of items
        \item Dictionaries are unordered collections of key-value pairs
        \item Lists are indexed by integers
        \item Dictionaries are indexed by keys
    \end{itemize}
\end{frame}

\begin{frame}
    \centering
    \frametitle{Combining Lists and Dictionaries}
    \begin{itemize}
        \setlength{\itemsep}{2em}
        \item You can combine lists and dictionaries to create complex data structures
        \item You can nest lists and dictionaries to create hierarchical data structures
    \end{itemize}
\end{frame}

\begin{frame}[fragile]
    \frametitle{Combining Lists and Dictionaries}
    \begin{lstlisting}[language=Python]
person = {
    "name": "John",
    "age": 36,
    "friends": ["Jane", "Jack"]
}
print(person["name"])
print(person["age"])
print(person["friends"])
    \end{lstlisting}
\end{frame}

\begin{frame}
    \centering
    \frametitle{Reading and Writing Files}
    \begin{itemize}
        \setlength{\itemsep}{2em}
        \item Python allows you to read and write files
        \item It allows you to open files in read mode, write mode, or append mode
        \item It allows you to read files line by line or all at once
    \end{itemize}
\end{frame}

\begin{frame}[fragile]
    \frametitle{Reading and Writing Files}
    \begin{lstlisting}[language=Python]
with open("data.txt", "w") as file:
    file.write("Hello, world!")

with open("data.txt", "r") as file:
    data = file.read()
    print(data)
    \end{lstlisting}
\end{frame}

\begin{frame}
    \centering
    \frametitle{NumPy}
    \begin{itemize}
        \setlength{\itemsep}{2em}
        \item NumPy is a library for numerical computing
        \item It provides support for arrays and matrices
        \item It allows you to perform mathematical operations on arrays and matrices
        \item It is the foundation of many other libraries
    \end{itemize}
\end{frame}

\begin{frame}[fragile]
    \frametitle{NumPy}
    \begin{lstlisting}[language=Python]
import numpy as np

a = np.array([1, 2, 3])
b = np.array([4, 5, 6])
c = a * b
print(c)

d = np.dot(a, b)
print(d)
    \end{lstlisting}
\end{frame}

\begin{frame}
    \centering
    \frametitle{Pandas}
    \begin{itemize}
        \setlength{\itemsep}{2em}
        \item Pandas is a library for data manipulation and analysis
        \item It provides support for data structures like Series and DataFrame
        \item It allows you to read and write data from various sources
        \item It is built on top of NumPy
    \end{itemize}
\end{frame}

\begin{frame}
    \centering
    \frametitle{Pandas IO}
    \begin{itemize}
        \setlength{\itemsep}{2em}
        \item Pandas allows you to read and write data from various sources
        \item It allows you to read and write data from CSV files, Excel files, SQL databases, and APIs
        \item It allows you to read and write data from URLs, HTML tables, and clipboard
    \end{itemize}
\end{frame}

\begin{frame}[fragile]
    \frametitle{Pandas IO}
    \begin{lstlisting}[language=Python]
import pandas as pd

data = pd.read_csv("data.csv")
print(data)

data.to_csv("data.csv", index=False)
    \end{lstlisting}
\end{frame}

\begin{frame}
    \centering
    \frametitle{Manipulating Dataframes}
    \begin{itemize}
        \setlength{\itemsep}{2em}
        \item Pandas allows you to manipulate dataframes
        \item It allows you to filter, sort, group, and aggregate data
        \item It allows you to merge, join, and concatenate data
        \item It allows you to reshape, pivot, and melt data
    \end{itemize}
\end{frame}

\begin{frame}[fragile]
    \frametitle{Manipulating Dataframes}
    \begin{lstlisting}[language=Python]
import pandas as pd

data = pd.read_csv("age.csv")
data = data[data["age"] > 30]
data = data.sort_values("age")
data['rank'] = data['age'].rank()
data.loc['total'] = data.sum()
print(data)
    \end{lstlisting}
\end{frame}

\begin{frame}
    \centering
    \frametitle{pd.apply()}
    \begin{itemize}
        \setlength{\itemsep}{2em}
        \item Pandas allows you to apply functions to dataframes
        \item It allows you to apply functions to rows, columns, or cells
        \item It allows you to apply lambda functions, user-defined functions, or built-in functions
    \end{itemize}
\end{frame}

\begin{frame}[fragile]
    \frametitle{pd.apply()}
    \begin{lstlisting}[language=Python]
import pandas as pd

data = pd.read_csv("age.csv")
data['age'] = data['age'].apply(lambda x: x + 1)
print(data)
    \end{lstlisting}
\end{frame}

\begin{frame}
    \centering
    \frametitle{Merge and Join}
    \begin{itemize}
        \setlength{\itemsep}{2em}
        \item Pandas allows you to merge and join dataframes
        \item It allows you to merge dataframes on columns or indices
        \item It allows you to merge dataframes using inner, outer, left, or right joins
    \end{itemize}
\end{frame}

\begin{frame}[fragile]
    \frametitle{Merge and Join}
    \begin{lstlisting}[language=Python]
import pandas as pd

data1 = pd.read_csv("age.csv")
data2 = pd.read_csv("blood_type.csv")
data = pd.merge(data1, data2, on="name")
print(data)
    \end{lstlisting}
\end{frame}

\begin{frame}
    \centering
    \frametitle{Null Values}
    \begin{itemize}
        \setlength{\itemsep}{2em}
        \item Pandas allows you to handle null values
        \item It allows you to drop null values, fill null values, or interpolate null values
        \item It allows you to check for null values, count null values, or filter null values
    \end{itemize}
\end{frame}

\begin{frame}[fragile]
    \frametitle{Null Values}
    \begin{lstlisting}[language=Python]
import pandas as pd

data = pd.read_csv("temp.csv")
data = data.dropna()
data = data.fillna(0)
data = data.interpolate()
print(data)
    \end{lstlisting}
\end{frame}

\begin{frame}
    \centering
    \frametitle{Rules of Thumb for Missing Data}
    \begin{itemize}
        \setlength{\itemsep}{2em}
        \item If the missing data is random, drop the rows
        \item If the missing data is systematic, fill the missing values
        \item If the missing data is time-dependent, interpolate the missing values
    \end{itemize}
\end{frame}

\end{document}