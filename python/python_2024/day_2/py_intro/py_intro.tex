\documentclass[serif, 9pt, aspectratio=32]{beamer} 
\usetheme{Darmstadt}

\usepackage{appendixnumberbeamer}
\usepackage{booktabs}
\usepackage[scale=2]{ccicons}
\usepackage{pgfplots}
\usepgfplotslibrary{dateplot}
\usepackage{xspace}
\usepackage{tikz}
\usepackage{hyperref}
\usepackage{xcolor}
\usepackage{listings}
\usetikzlibrary{shapes, arrows}
\newcommand{\themename}{\textbf{\textsc{metropolis}}\xspace}

\title{Introduction to Python}
\date{\today}
\author{Tan Sein Jone}
\institute{University of British Columbia}

\pgfplotsset{compat=1.18}
\setbeamertemplate{footline}[frame number]

\begin{document}

\maketitle

\begin{frame}{Table of contents}
    \setbeamertemplate{section in toc}[sections numbered]
    \tableofcontents[hideallsubsections]
\end{frame}

\section{Object Oriented Programming}

\begin{frame}
    \frametitle{Table of Contents}
    \setbeamertemplate{section in toc}[sections numbered]
    \tableofcontents[currentsection]
\end{frame}

\begin{frame}
    \centering
    \frametitle{Classes and Objects}
    \begin{itemize}
        \setlength{\itemsep}{2em}
        \item A class is a blueprint for creating object
        \item Classes define the properties and behaviours of objects
        \item Objects have attributes and methods
        \item Attributes are variables that store data
        \item Methods are functions that perform actions
    \end{itemize}
\end{frame}

\begin{frame}
    \centering
    \frametitle{Examples of Classes in Pandas}
    \begin{itemize}
        \setlength{\itemsep}{2em}
        \item \texttt{pd.DataFrame} is a class that represents a two-dimensional table of data
        \item \texttt{pd.Series} is a class that represents a one-dimensional array of data
        \item \texttt{pd.Index} is a class that represents an index of data
    \end{itemize}
\end{frame}

\begin{frame}
    \centering
    \frametitle{Examples of Methods in Pandas}
    \begin{itemize}
        \setlength{\itemsep}{2em}
        \item \texttt{pd.DataFrame.head()} returns the first $n$ rows of a dataframe
        \item \texttt{pd.DataFrame.tail()} returns the last $n$ rows of a dataframe
        \item \texttt{pd.DataFrame.describe()} returns the summary statistics of a dataframe
    \end{itemize}
\end{frame}

\begin{frame}
    \centering
    \frametitle{Attributes}
    \begin{itemize}
        \setlength{\itemsep}{2em}
        \item Attributes are variables that store data
        \item They are defined in the \texttt{\_\_init\_\_} method
        \item They are accessed using the dot operator
    \end{itemize}
\end{frame}

\begin{frame}[fragile]
    \frametitle{Attributes}
    \begin{lstlisting}[language=Python]
class Person:
    def __init__(self, name, age):
        self.name = name
        self.age = age
    \end{lstlisting}
\end{frame}

\begin{frame}
    \centering
    \frametitle{Breakdown}
    \begin{itemize}
        \setlength{\itemsep}{2em}
        \item \texttt{class Person:} defines a class named \texttt{Person}
        \item \texttt{def \_\_init\_\_(self, name, age):} defines a method named \texttt{\_\_init\_\_} that initializes the object
        \item \texttt{self} is a reference to the object itself
        \item \texttt{self.name} and \texttt{self.age} are attributes of the object
    \end{itemize}
\end{frame}

\begin{frame}[fragile]
    \frametitle{Attributes}
    \begin{lstlisting}[language=Python]
person = Person("John", 36)
print(person.name)
print(person.age)
    \end{lstlisting}
\end{frame}

\begin{frame}
    \centering
    \frametitle{Methods}
    \begin{itemize}
        \setlength{\itemsep}{2em}
        \item Methods are functions that perform actions
        \item They are defined in the class
        \item They are accessed using the dot operator
    \end{itemize}
\end{frame}

\begin{frame}[fragile]
    \frametitle{Methods}
    \begin{lstlisting}[language=Python]
class Person:
    def __init__(self, name, age):
        self.name = name
        self.age = age

    def greet(self):
        print(f"Hello, my name is {self.name}")
    \end{lstlisting}
\end{frame}

\begin{frame}
    \centering
    \frametitle{Breakdown}
    \begin{itemize}
        \setlength{\itemsep}{2em}
        \item \texttt{class Person:} defines a class named \texttt{Person}
        \item \texttt{def \_\_init\_\_(self, name, age):} defines a method named \texttt{\_\_init\_\_} that initializes the object
        \item \texttt{self} is a reference to the object itself
        \item \texttt{self.name} and \texttt{self.age} are attributes of the object
        \item \texttt{def greet(self):} defines a method named \texttt{greet} that prints a greeting
    \end{itemize}
\end{frame}

\begin{frame}[fragile]
    \frametitle{Methods}
    \begin{lstlisting}[language=Python]
person = Person("John", 36)
person.greet()
    \end{lstlisting}
\end{frame}

\begin{frame}
    \centering
    \frametitle{Breakdown}
    \begin{itemize}
        \setlength{\itemsep}{2em}
        \item \texttt{person = Person("John", 36)} creates an object of class \texttt{Person}
        \item \texttt{person.greet()} calls the \texttt{greet} method of the object
    \end{itemize}
\end{frame}

\begin{frame}
    \centering
    \frametitle{Benefits of OOP}
    \begin{itemize}
        \setlength{\itemsep}{2em}
        \item Encapsulation
        \item Inheritance
        \item Polymorphism
    \end{itemize}
\end{frame}

\begin{frame}
    \centering
    \frametitle{Encapsulation}
    \begin{itemize}
        \setlength{\itemsep}{2em}
        \item Encapsulation is the bundling of data and methods that operate on the data
        \item It restricts access to some of the object's components
        \item It prevents the accidental modification of data
    \end{itemize}
\end{frame}

\begin{frame}
    \centering
    \frametitle{Inheritance}
    \begin{itemize}
        \setlength{\itemsep}{2em}
        \item Inheritance is the mechanism of basing a class upon another class
        \item It allows a class to inherit attributes and methods from another class
        \item It allows a class to override methods of another class
    \end{itemize}
\end{frame}

\begin{frame}
    \centering
    \frametitle{Polymorphism}
    \begin{itemize}
        \setlength{\itemsep}{2em}
        \item Polymorphism is the ability to present the same interface for different data types
        \item It allows a function to accept different data types
        \item It allows a class to override methods of another class
    \end{itemize}
\end{frame}

\begin{frame}
    \centering
    \frametitle{Do I Need to Know How to Write Classes?}
    \begin{itemize}
        \setlength{\itemsep}{2em}
        \item For the purposes of the program, probably not
        \item Why did I cover this topic? Because it is important to understand how Python works under the hood
        \item When you call a method in Pandas, you are calling a method of a class
        \item When you create a dataframe in Pandas, you are creating an object of a class
        \item This understanding will make it much easier to debug and troubleshoot your code
    \end{itemize}
\end{frame}

\begin{frame}
    \centering
    \frametitle{Summary}
    \begin{itemize}
        \setlength{\itemsep}{2em}
        \item Classes are blueprints for creating objects
        \item Objects have attributes and methods
        \item Attributes are variables that store data
        \item Methods are functions that perform actions
        \item Encapsulation is the bundling of data and methods
        \item Inheritance is the mechanism of basing a class upon another class
        \item Polymorphism is the ability to present the same interface for different data types
    \end{itemize}
\end{frame}

\section{Pythonics}

\begin{frame}
    \frametitle{Table of Contents}
    \setbeamertemplate{section in toc}[sections numbered]
    \tableofcontents[currentsection]
\end{frame}

\begin{frame}
    \centering
    \frametitle{Pythonics}
    \begin{itemize}
        \setlength{\itemsep}{2em}
        \item Pythonic code is code that follows the conventions of the Python language
        \item It is code that is clean, readable, and maintainable
        \item It is code that is idiomatic and expressive
    \end{itemize}
\end{frame}

\begin{frame}
    \centering
    \frametitle{Zen of Python}
    \begin{itemize}
        \setlength{\itemsep}{2em}
        \item The Zen of Python is a collection of aphorisms that capture the philosophy of Python
        \item It is a set of guiding principles for writing computer programs
        \item It is a set of rules for writing Pythonic code
    \end{itemize}
\end{frame}

\begin{frame}
    \centering
    \frametitle{Zen of Python}
    \begin{itemize}
        \setlength{\itemsep}{2em}
        \item Beautiful is better than ugly
        \item Explicit is better than implicit
        \item Simple is better than complex
        \item Complex is better than complicated
        \item Readability counts
        \item There should be one-- and preferably only one --obvious way to do it
        \item Now is better than never
        \item Although never is often better than right now
        \item If the implementation is hard to explain, it's a bad idea
        \item If the implementation is easy to explain, it may be a good idea
    \end{itemize}
\end{frame}

\begin{frame}
    \centering
    \frametitle{Value Swapping and Multiple Assignment}
    \begin{itemize}
        \setlength{\itemsep}{2em}
        \item Python allows you to swap the values of two variables in a single line
        \item It also allows you to assign multiple values to multiple variables in a single line
    \end{itemize}
\end{frame}

\begin{frame}[fragile]
    \frametitle{Value Swapping and Multiple Assignment}
    \begin{lstlisting}[language=Python]
a = 1
b = 2
a, b = b, a
print(a, b)
    \end{lstlisting}
\end{frame}

\begin{frame}
    \centering
    \frametitle{Breakdown}
    \begin{itemize}
        \setlength{\itemsep}{2em}
        \item \texttt{a, b = b, a} swaps the values of \texttt{a} and \texttt{b}
        \item \texttt{print(a, b)} prints the values of \texttt{a} and \texttt{b}
        \item The output is \texttt{2 1}
    \end{itemize}
\end{frame}

\begin{frame}
    \centering
    \frametitle{List Slicing}
    \begin{itemize}
        \setlength{\itemsep}{2em}
        \item Python allows you to slice lists using the slice operator
        \item It allows you to slice lists using the start, stop, and step arguments
        \item It allows you to slice lists using negative indices
    \end{itemize}
\end{frame}

\begin{frame}[fragile]
    \frametitle{List Slicing}
    \begin{lstlisting}[language=Python]
numbers = [1, 2, 3, 4, 5]
print(numbers[1:3])
print(numbers[::2])
print(numbers[::-1])
    \end{lstlisting}
\end{frame}

\begin{frame}
    \centering
    \frametitle{Breakdown}
    \begin{itemize}
        \setlength{\itemsep}{2em}
        \item \texttt{print(numbers[1:3])} slices the list from index 1 to index 3
        \item \texttt{print(numbers[::2])} slices the list with a step of 2
        \item \texttt{print(numbers[::-1])} slices the list in reverse order
        \item The output is \texttt{[2, 3], [1, 3, 5], [5, 4, 3, 2, 1]}
    \end{itemize}
\end{frame}

\begin{frame}
    \centering
    \frametitle{Passing Multiple Arguments}
    \begin{itemize}
        \setlength{\itemsep}{2em}
        \item Python allows you to pass multiple arguments to a function
        \item It also allows you to pass keyword arguments to a function
    \end{itemize}
\end{frame}

\begin{frame}[fragile]
    \frametitle{Passing Multiple Arguments}
    \begin{lstlisting}[language=Python]
def greet(*names):
    for name in names:
        print(f"Hello, {name}")

greet("John", "Jane", "Jack")
    \end{lstlisting}
\end{frame}

\begin{frame}
    \centering
    \frametitle{Breakdown}
    \begin{itemize}
        \setlength{\itemsep}{2em}
        \item \texttt{def greet(*names):} defines a function named \texttt{greet} that takes multiple arguments
        \item \texttt{for name in names:} iterates over the arguments
        \item \texttt{print(f"Hello, {name}")} prints a greeting for each argument
        \item The output is \texttt{Hello, John, Hello, Jane, Hello, Jack}
    \end{itemize}
\end{frame}

\begin{frame}
    \centering
    \frametitle{List Comprehension}
    \begin{itemize}
        \setlength{\itemsep}{2em}
        \item List comprehension is a concise way to create lists
        \item It allows you to create lists using a single line of code
        \item It is more readable and expressive than traditional loops
    \end{itemize}
\end{frame}

\begin{frame}[fragile]
    \frametitle{List Comprehension}
    \begin{lstlisting}[language=Python]
squares = [x ** 2 for x in range(10)]
print(squares)
    \end{lstlisting}
\end{frame}

\begin{frame}
    \centering
    \frametitle{Breakdown}
    \begin{itemize}
        \setlength{\itemsep}{2em}
        \item \texttt{squares = [x ** 2 for x in range(10)]} creates a list of squares
        \item \texttt{print(squares)} prints the list of squares
        \item The output is \texttt{[0, 1, 4, 9, 16, 25, 36, 49, 64, 81]}
    \end{itemize}
\end{frame}

\begin{frame}
    \centering
    \frametitle{List Comprehension vs Loops}
    \begin{itemize}
        \setlength{\itemsep}{2em}
        \item It is possible to achieve the same result using a loop
        \item However, list comprehension is more concise and expressive
        \item It is also more readable and maintainable
        \item It is the preferred way to create lists in Python
        \item If the list comprehension is too complex, use a loop instead
    \end{itemize}
\end{frame}

\begin{frame}
    \centering
    \frametitle{Lambda Functions}
    \begin{itemize}
        \setlength{\itemsep}{2em}
        \item Lambda functions are anonymous functions
        \item They are defined using the \texttt{lambda} keyword
        \item They are used to create small, one-line functions
    \end{itemize}
\end{frame}

\begin{frame}[fragile]
    \frametitle{Lambda Functions}
    \begin{lstlisting}[language=Python]
add = lambda x, y: x + y
print(add(1, 2))
    \end{lstlisting}
\end{frame}

\begin{frame}
    \centering
    \frametitle{Breakdown}
    \begin{itemize}
        \setlength{\itemsep}{2em}
        \item \texttt{add = lambda x, y: x + y} defines a lambda function that adds two numbers
        \item \texttt{print(add(1, 2))} calls the lambda function with arguments 1 and 2
        \item The output is \texttt{3}
    \end{itemize}
\end{frame}

\begin{frame}
    \centering
    \frametitle{A Note on Indentation}
    \begin{itemize}
        \setlength{\itemsep}{2em}
        \item Python uses indentation to define blocks of code
        \item It uses whitespace to delimit code
    \end{itemize}
\end{frame}

\begin{frame}
    \centering
    \frametitle{A Note on Indentation}
    \begin{itemize}
        \setlength{\itemsep}{2em}
        \item Indentation is important in Python
        \item It is used to define the scope of code
        \item It is used to group statements together
    \end{itemize}
\end{frame}

\begin{frame}
    \centering
    \frametitle{Summary}
    \begin{itemize}
        \setlength{\itemsep}{2em}
        \item Pythonic code is code that follows the conventions of the Python language
        \item It is code that is clean, readable, and maintainable
        \item It is code that is idiomatic and expressive
        \item The Zen of Python is a collection of aphorisms that capture the philosophy of Python
        \item It is a set of guiding principles for writing computer programs
        \item It is a set of rules for writing Pythonic code
    \end{itemize}
\end{frame}

\section{Data Handling}

\begin{frame}
    \frametitle{Table of Contents}
    \setbeamertemplate{section in toc}[sections numbered]
    \tableofcontents[currentsection]
\end{frame}

\begin{frame}
    \centering
    \frametitle{Data Handling}
    \begin{itemize}
        \setlength{\itemsep}{2em}
        \item Data handling is the process of managing data
        \item It involves reading, writing, and processing data
        \item It involves working with files, databases, and APIs
    \end{itemize}
\end{frame}

\begin{frame}
    \centering
    \frametitle{Lists and Dictionaries}
    \begin{itemize}
        \setlength{\itemsep}{2em}
        \item Lists are ordered collections of items
        \item Dictionaries are unordered collections of key-value pairs
        \item Lists are indexed by integers
        \item Dictionaries are indexed by keys
    \end{itemize}
\end{frame}

\begin{frame}
    \centering
    \frametitle{Combining Lists and Dictionaries}
    \begin{itemize}
        \setlength{\itemsep}{2em}
        \item You can combine lists and dictionaries to create complex data structures
        \item You can nest lists and dictionaries to create hierarchical data structures
    \end{itemize}
\end{frame}

\begin{frame}[fragile]
    \frametitle{Combining Lists and Dictionaries}
    \begin{lstlisting}[language=Python]
person = {
    "name": "John",
    "age": 36,
    "friends": ["Jane", "Jack"]
}
print(person["name"])
print(person["age"])
print(person["friends"])
    \end{lstlisting}
\end{frame}

\begin{frame}
    \centering
    \frametitle{Breakdown}
    \begin{itemize}
        \setlength{\itemsep}{2em}
        \item \texttt{person = \{"name": "John", "age": 36, "friends": ["Jane", "Jack"]\}} creates a dictionary
        \item \texttt{print(person["name"])} prints the value of the key \texttt{"name"}
        \item \texttt{print(person["age"])} prints the value of the key \texttt{"age"}
        \item \texttt{print(person["friends"])} prints the value of the key \texttt{"friends"}
        \item The output is \texttt{John, 36, ["Jane", "Jack"]}
    \end{itemize}
\end{frame}

\begin{frame}
    \centering
    \frametitle{Reading and Writing Files}
    \begin{itemize}
        \setlength{\itemsep}{2em}
        \item Python allows you to read and write files
        \item It allows you to open files in read mode, write mode, or append mode
        \item It allows you to read files line by line or all at once
    \end{itemize}
\end{frame}

\begin{frame}[fragile]
    \frametitle{Reading and Writing Files}
    \begin{lstlisting}[language=Python]
with open("data.txt", "w") as file:
    file.write("Hello, world!")

with open("data.txt", "r") as file:
    data = file.read()
    print(data)
    \end{lstlisting}
\end{frame}

\begin{frame}
    \centering
    \frametitle{Reading and Writing Files}
    \begin{itemize}
        \setlength{\itemsep}{2em}
        \item Python allows you to read and write files
        \item It allows you to open files in read mode, write mode, or append mode
        \item It allows you to read files line by line or all at once
    \end{itemize}
\end{frame}

\begin{frame}
    \centering
    \frametitle{NumPy}
    \begin{itemize}
        \setlength{\itemsep}{2em}
        \item NumPy is a library for numerical computing
        \item It provides support for arrays and matrices
        \item It allows you to perform mathematical operations on arrays and matrices
        \item It is the foundation of many other libraries
    \end{itemize}
\end{frame}

\begin{frame}[fragile]
    \frametitle{NumPy}
    \begin{lstlisting}[language=Python]
import numpy as np

a = np.array([1, 2, 3])
b = np.array([4, 5, 6])
c = a * b
print(c)

d = np.dot(a, b)
print(d)
    \end{lstlisting}
\end{frame}

\begin{frame}
    \centering
    \frametitle{Breakdown}
    \begin{itemize}
        \setlength{\itemsep}{2em}
        \item \texttt{a = np.array([1, 2, 3])} creates an array \texttt{a}
        \item \texttt{b = np.array([4, 5, 6])} creates an array \texttt{b}
        \item \texttt{c = a * b} multiplies the arrays \texttt{a} and \texttt{b} element-wise
        \item \texttt{print(c)} prints the result of the multiplication
        \item \texttt{d = np.dot(a, b)} computes the dot product of the arrays \texttt{a} and \texttt{b}
        \item \texttt{print(d)} prints the result of the dot product
    \end{itemize}
\end{frame}

\begin{frame}
    \centering
    \frametitle{Pandas}
    \begin{itemize}
        \setlength{\itemsep}{2em}
        \item Pandas is a library for data manipulation and analysis
        \item It provides support for data structures like Series and DataFrame
        \item It allows you to read and write data from various sources
        \item It is built on top of NumPy
    \end{itemize}
\end{frame}

\begin{frame}
    \centering
    \frametitle{Pandas IO}
    \begin{itemize}
        \setlength{\itemsep}{2em}
        \item Pandas allows you to read and write data from various sources
        \item It allows you to read and write data from CSV files, Excel files, SQL databases, and APIs
        \item It allows you to read and write data from URLs, HTML tables, and clipboard
    \end{itemize}
\end{frame}

\begin{frame}[fragile]
    \frametitle{Pandas IO}
    \begin{lstlisting}[language=Python]
import pandas as pd

data = pd.read_csv("data.csv")
print(data)

data.to_csv("data.csv", index=False)
    \end{lstlisting}
\end{frame}

\begin{frame}
    \centering
    \frametitle{Breakdown}
    \begin{itemize}
        \setlength{\itemsep}{2em}
        \item \texttt{data = pd.read\_csv("data.csv")} reads a CSV file into a dataframe
        \item \texttt{print(data)} prints the dataframe
        \item \texttt{data.to\_csv("data.csv", index=False)} writes the dataframe to a CSV file
    \end{itemize}
\end{frame}

\begin{frame}
    \centering
    \frametitle{A Note on File Pathing in Linux}
    \begin{itemize}
        \setlength{\itemsep}{2em}
        \item In Linux, the Python Interpreter knows to look in the current directory for files
        \item If you are running Python in a different directory, you can use a relative path to the file
        \item If you are running Python in a different directory, you can use an absolute path to the file
    \end{itemize}
\end{frame}

\begin{frame}
    \centering
    \frametitle{Manipulating Dataframes}
    \begin{itemize}
        \setlength{\itemsep}{2em}
        \item Pandas allows you to manipulate dataframes
        \item It allows you to filter, sort, group, and aggregate data
        \item It allows you to merge, join, and concatenate data
        \item It allows you to reshape, pivot, and melt data
    \end{itemize}
\end{frame}

\begin{frame}[fragile]
    \frametitle{Manipulating Dataframes}
    \begin{lstlisting}[language=Python]
import pandas as pd

data = pd.read_csv("age.csv")
data = data[data["age"] > 30]
data = data.sort_values("age")
data['rank'] = data['age'].rank()
data.loc['total'] = data.sum()
print(data)
    \end{lstlisting}
\end{frame}

\begin{frame}
    \centering
    \frametitle{Breakdown}
    \begin{itemize}
        \setlength{\itemsep}{2em}
        \item \texttt{data = pd.read\_csv("age.csv")} reads a CSV file into a dataframe
        \item \texttt{data = data[data["age"] > 30]} filters the dataframe by age
        \item \texttt{data = data.sort\_values("age")} sorts the dataframe by age
        \item \texttt{data['rank'] = data['age'].rank()} ranks the dataframe by age
        \item \texttt{data.loc['total'] = data.sum()} sums the dataframe
        \item \texttt{print(data)} prints the dataframe
    \end{itemize}
\end{frame}

\begin{frame}
    \centering
    \frametitle{pd.apply()}
    \begin{itemize}
        \setlength{\itemsep}{2em}
        \item Pandas allows you to apply functions to dataframes
        \item It allows you to apply functions to rows, columns, or cells
        \item It allows you to apply lambda functions, user-defined functions, or built-in functions
    \end{itemize}
\end{frame}

\begin{frame}[fragile]
    \frametitle{pd.apply()}
    \begin{lstlisting}[language=Python]
import pandas as pd

data = pd.read_csv("age.csv")
data['age'] = data['age'].apply(lambda x: x + 1)
print(data)
    \end{lstlisting}
\end{frame}

\begin{frame}
    \centering
    \frametitle{Breakdown}
    \begin{itemize}
        \setlength{\itemsep}{2em}
        \item \texttt{data = pd.read\_csv("age.csv")} reads a CSV file into a dataframe
        \item \texttt{data['age'] = data['age'].apply(lambda x: x + 1)} applies a lambda function to the age column
        \item \texttt{print(data)} prints the dataframe
    \end{itemize}
\end{frame}

\begin{frame}
    \centering
    \frametitle{Merge and Join}
    \begin{itemize}
        \setlength{\itemsep}{2em}
        \item Pandas allows you to merge and join dataframes
        \item It allows you to merge dataframes on columns or indices
        \item It allows you to merge dataframes using inner, outer, left, or right joins
    \end{itemize}
\end{frame}

\begin{frame}[fragile]
    \frametitle{Merge and Join}
    \begin{lstlisting}[language=Python]
import pandas as pd

data1 = pd.read_csv("age.csv")
data2 = pd.read_csv("blood_type.csv")
data = pd.merge(data1, data2, on="name")
print(data)
    \end{lstlisting}
\end{frame}

\begin{frame}
    \centering
    \frametitle{Breakdown}
    \begin{itemize}
        \setlength{\itemsep}{2em}
        \item \texttt{data1 = pd.read\_csv("age.csv")} reads a CSV file into a dataframe
        \item \texttt{data2 = pd.read\_csv("blood\_type.csv")} reads a CSV file into a dataframe
        \item \texttt{data = pd.merge(data1, data2, on="name")} merges the dataframes on the \texttt{"name"} column
        \item \texttt{print(data)} prints the merged dataframe
    \end{itemize}
\end{frame}

\begin{frame}
    \centering
    \frametitle{Null Values}
    \begin{itemize}
        \setlength{\itemsep}{2em}
        \item Pandas allows you to handle null values
        \item It allows you to drop null values, fill null values, or interpolate null values
        \item It allows you to check for null values, count null values, or filter null values
    \end{itemize}
\end{frame}

\begin{frame}[fragile]
    \frametitle{Null Values}
    \begin{lstlisting}[language=Python]
import pandas as pd

data = pd.read_csv("temp.csv")
data = data.dropna()
data = data.fillna(0)
data = data.interpolate()
print(data)
    \end{lstlisting}
\end{frame}

\begin{frame}
    \centering
    \frametitle{Breakdown}
    \begin{itemize}
        \setlength{\itemsep}{2em}
        \item \texttt{data = pd.read\_csv("temp.csv")} reads a CSV file into a dataframe
        \item \texttt{data = data.dropna()} drops null values from the dataframe
        \item \texttt{data = data.fillna(0)} fills null values with 0
        \item \texttt{data = data.interpolate()} interpolates null values
        \item \texttt{print(data)} prints the dataframe
    \end{itemize}
\end{frame}

\begin{frame}
    \centering
    \frametitle{Rules of Thumb for Missing Data}
    \begin{itemize}
        \setlength{\itemsep}{2em}
        \item If the missing data is random, drop the rows
        \item If the missing data is systematic, fill the missing values
        \item If the missing data is time-dependent, interpolate the missing values
    \end{itemize}
\end{frame}

\begin{frame}
    \centering
    \frametitle{Best Practices for Missing Data}
    \begin{itemize}
        \setlength{\itemsep}{2em}
        \item Regardless of what you do with missing data, always document your decisions
        \item Always check for missing data before performing any analysis
        \item Always check for missing data after performing any analysis
    \end{itemize}
\end{frame}

\begin{frame}
    \centering
    \frametitle{Summary}
    \begin{itemize}
        \setlength{\itemsep}{2em}
        \item Data handling is the process of managing data
        \item It involves reading, writing, and processing data
        \item It involves working with files, databases, and APIs
        \item Lists are ordered collections of items
        \item Dictionaries are unordered collections of key-value pairs
        \item NumPy is a library for numerical computing
        \item Pandas is a library for data manipulation and analysis
    \end{itemize}
\end{frame}

\end{document}