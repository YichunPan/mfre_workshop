\documentclass[serif, 9pt, aspectratio=32]{beamer} 
\usetheme{Darmstadt}

\usepackage{appendixnumberbeamer}
\usepackage{booktabs}
\usepackage[scale=2]{ccicons}
\usepackage{pgfplots}
\usepgfplotslibrary{dateplot}
\usepackage{xspace}
\usepackage{tikz}
\usepackage{hyperref}
\usepackage{xcolor}
\usepackage{listings}
\usetikzlibrary{shapes, arrows}
\newcommand{\themename}{\textbf{\textsc{metropolis}}\xspace}

\title{Introduction to R}
\date{\today}
\author{Tan Sein Jone}
\institute{University of British Columbia}

\pgfplotsset{compat=1.18}
\setbeamertemplate{footline}[frame number]

\begin{document}

\maketitle

\begin{frame}{Table of contents}
    \setbeamertemplate{section in toc}[sections numbered]
    \tableofcontents[hideallsubsections]
\end{frame}

\section{Data Visualization}

\begin{frame}
    \frametitle{Table of Contents}
    \setbeamertemplate{section in toc}[sections numbered]
    \tableofcontents[currentsection]
\end{frame}

\begin{frame}
    \centering
    \frametitle{Data Visualization}
    \begin{itemize}
        \setlength{\itemsep}{2em}
        \item Data visualization is the process of representing data graphically.
        \item R has a wide range of functions and packages that make data visualization easier.
        \item The most common types of data visualizations are scatter plots, bar charts, and line charts.
    \end{itemize}
\end{frame}

\begin{frame}
    \centering
    \frametitle{Base Plotting}
    \begin{itemize}
        \setlength{\itemsep}{2em}
        \item Base plotting is the default plotting system in R.
        \item Base plotting is simple and easy to use.
        \item Base plotting is good for creating simple plots.
    \end{itemize}
\end{frame}

\begin{frame}[fragile]
    \begin{lstlisting}
        x <- c(1, 2, 3, 4, 5)
        y <- c(2, 4, 6, 8, 10)

        plot(x, y)
    \end{lstlisting}
\end{frame}

\begin{frame}
    \centering
    \frametitle{ggplot2}
    \begin{itemize}
        \setlength{\itemsep}{2em}
        \item ggplot2 is a popular plotting package in R.
        \item ggplot2 is based on the grammar of graphics.
        \item ggplot2 is good for creating complex plots.
    \end{itemize}
\end{frame}

\begin{frame}[fragile]
    \begin{lstlisting}
        library(ggplot2)

        data <- data.frame(
            x = c(1, 2, 3, 4, 5),
            y = c(2, 4, 6, 8, 10)
        )

        ggplot(data, aes(x = x, y = y)) + geom_point()
    \end{lstlisting}
\end{frame}

\begin{frame}
    \centering
    \frametitle{Plotly}
    \begin{itemize}
        \setlength{\itemsep}{2em}
        \item Plotly is an interactive plotting package in R.
        \item Plotly is based on the Plotly.js library.
        \item Plotly is good for creating interactive plots.
    \end{itemize}
\end{frame}

\begin{frame}[fragile]
    \begin{lstlisting}
        library(plotly)

        data <- data.frame(
            x = c(1, 2, 3, 4, 5),
            y = c(2, 4, 6, 8, 10)
        )

        plot_ly(data, x = ~x, y = ~y, type = "scatter", mode = "markers")
    \end{lstlisting}
\end{frame}

\section{Tips for Tidyverse}

\begin{frame}
    \frametitle{Table of Contents}
    \setbeamertemplate{section in toc}[sections numbered]
    \tableofcontents[currentsection]
\end{frame}

\begin{frame}
    \centering
    \frametitle{Tips for Tidyverse}
    \begin{itemize}
        \setlength{\itemsep}{2em}
        \item The Tidyverse is a collection of R packages designed for data science.
        \item The Tidyverse is based on the principles of tidy data.
        \item The Tidyverse is good for data manipulation and visualization.
    \end{itemize}
\end{frame}

\begin{frame}
    \centering
    \frametitle{Pipes}
    \begin{itemize}
        \setlength{\itemsep}{2em}
        \item Pipes are a way to chain R functions together.
        \item Pipes make it easy to read and write code.
        \item Pipes are good for data manipulation and visualization.
    \end{itemize}
\end{frame}

\begin{frame}[fragile]
    \begin{lstlisting}
        library(dplyr)

        data <- data.frame(
            x = c(1, 2, 3, 4, 5),
            y = c(2, 4, 6, 8, 10)
        )

        data %>%
            filter(x > 2) %>%
            ggplot(aes(x = x, y = y)) +
            geom_point()
    \end{lstlisting}
\end{frame}

\begin{frame}
    \centering
    \frametitle{Tibbles}
    \begin{itemize}
        \setlength{\itemsep}{2em}
        \item Tibbles are a modern version of data frames.
        \item Tibbles are easier to read and write than data frames.
        \item Tibbles are good for data manipulation and visualization.
    \end{itemize}
\end{frame}

\begin{frame}[fragile]
    \begin{lstlisting}
        library(dplyr)

        data <- tibble(
            x = c(1, 2, 3, 4, 5),
            y = c(2, 4, 6, 8, 10)
        )

        data %>%
            filter(x > 2) %>%
            ggplot(aes(x = x, y = y)) +
            geom_point()
    \end{lstlisting}
\end{frame}

\begin{frame}
    \centering
    \frametitle{Grouping}
    \begin{itemize}
        \setlength{\itemsep}{2em}
        \item Grouping is a way to split data into groups.
        \item Grouping is good for data manipulation and visualization.
        \item Grouping is good for summarizing data.
    \end{itemize}
\end{frame}

\begin{frame}[fragile]
    \begin{lstlisting}
        library(dplyr)

        data <- data.frame(
            x = c(1, 2, 3, 4, 5),
            y = c(2, 4, 6, 8, 10),
            group = c("A", "A", "B", "B", "B")
        )

        data %>%
            group_by(group) %>%
            summarize(mean_y = mean(y))
    \end{lstlisting}
\end{frame}

\section{Functions Best Practices}

\begin{frame}
    \frametitle{Table of Contents}
    \setbeamertemplate{section in toc}[sections numbered]
    \tableofcontents[currentsection]
\end{frame}

\begin{frame}
    \centering
    \frametitle{Functions Best Practices}
    \begin{itemize}
        \setlength{\itemsep}{2em}
        \item Functions are a way to organize code in R.
        \item Functions are good for code reuse and readability.
        \item Functions are good for data manipulation and visualization.
    \end{itemize}
\end{frame}

\begin{frame}
    \centering
    \frametitle{Function Basics}
    \begin{itemize}
        \setlength{\itemsep}{2em}
        \item A function is a block of code that performs a specific task.
        \item A function takes input, processes it, and returns output.
        \item A function is defined using the function keyword.
    \end{itemize}
\end{frame}

\begin{frame}[fragile]
    \begin{lstlisting}
        add <- function(x, y){
            return(x + y)
        }

        add(5, 3)
    \end{lstlisting}
\end{frame}

\begin{frame}
    \centering
    \frametitle{Function Arguments}
    \begin{itemize}
        \setlength{\itemsep}{2em}
        \item A function can take zero or more arguments.
        \item Arguments are the input values that a function uses to perform its task.
        \item Arguments can have default values.
    \end{itemize}
\end{frame}

\begin{frame}[fragile]
    \begin{lstlisting}
        add <- function(x, y = 0){
            return(x + y)
        }

        add(5, 3)
        add(5)
    \end{lstlisting}
\end{frame}

\begin{frame}
    \centering
    \frametitle{Higher Order Functions}
    \begin{itemize}
        \setlength{\itemsep}{2em}
        \item Higher-order functions are functions that can either take other functions as arguments or return them as results.
        \item This is possible because functions are first-class citizens.
        \item Higher-order functions allow us to abstract over actions, not just values.
    \end{itemize}
\end{frame}

\begin{frame}
    \centering
    \frametitle{Tips for Higher Order Functions and Modularity}
    \begin{itemize}
        \setlength{\itemsep}{2em}
        \item Higher-order functions are functions that can take other functions as arguments or return them as results.
        \item Higher-order functions allow us to abstract over actions, not just values.
        \item Higher-order functions are good for modularity and code reuse.
    \end{itemize}
\end{frame}

\begin{frame}[fragile]
    \begin{lstlisting}
        add <- function(x, y){
            return(x + y)
        }

        subtract <- function(x, y){
            return(x - y)
        }

        operate <- function(func, x, y){
            return(func(x, y))
        }

        operate(add, 5, 3)
        operate(subtract, 5, 3)
    \end{lstlisting}
\end{frame}

\begin{frame}[fragile]
    \begin{lstlisting}
        add <- function(x, y){
            return(x + y)
        }

        subtract <- function(x, y){
            return(x - y)
        }

        create_operator <- function(op){
            if(op == "add"){
                return(add)
            } else if(op == "subtract"){
                return(subtract)
            }
        }

        operator <- create_operator("add")
        operator(5, 3)
    \end{lstlisting}
\end{frame}

\end{document}