\documentclass[serif, 9pt, aspectratio=32]{beamer} 
\usetheme{Darmstadt}

\usepackage{appendixnumberbeamer}
\usepackage{booktabs}
\usepackage[scale=2]{ccicons}
\usepackage{pgfplots}
\usepgfplotslibrary{dateplot}
\usepackage{xspace}
\usepackage{tikz}
\usepackage{hyperref}
\usepackage{xcolor}
\usepackage{listings}
\usetikzlibrary{shapes, arrows}
\newcommand{\themename}{\textbf{\textsc{metropolis}}\xspace}

\title{Introduction to R}
\date{\today}
\author{Tan Sein Jone}
\institute{University of British Columbia}

\pgfplotsset{compat=1.18}
\setbeamertemplate{footline}[frame number]

\begin{document}

\maketitle

\begin{frame}{Table of contents}
    \setbeamertemplate{section in toc}[sections numbered]
    \tableofcontents[hideallsubsections]
\end{frame}

\section{Functional Programming}

\begin{frame}
    \frametitle{Table of Contents}
    \setbeamertemplate{section in toc}[sections numbered]
    \tableofcontents[currentsection]
\end{frame}

\begin{frame}
    \frametitle{Functional Programming}
    \begin{itemize}
        \item Functional programming is a programming paradigm that treats computation as the evaluation of mathematical functions and avoids changing-state and mutable data.
        \item It is a declarative type of programming style.
        \item It focuses on what to solve rather than how to solve.
        \item It uses expressions instead of statements.
        \item It is based on mathematical functions.
    \end{itemize}
\end{frame}

\begin{frame}
    \centering
    \frametitle{Immutability}
    \begin{itemize}
        \setlength{\itemsep}{2em}
        \item In functional programming, data is immutable.
        \item This means that once a value is assigned to a variable, it should not be changed.
        \item This makes it easier to reason about the code and prevents bugs caused by side effects.
    \end{itemize}
\end{frame}

\begin{frame}
    \centering
    \frametitle{Pure Functions}
    \begin{itemize}
        \setlength{\itemsep}{2em}
        \item A pure function is a function where the output value is determined by its input values, without observable side effects.
        \item This is how functions in math work: Math.cos(x) will, for the same value of x, always return the same result.
        \item Pure functions are easier to reason about and test.
    \end{itemize}
\end{frame}

\begin{frame}[fragile]
    \begin{lstlisting}
        pure_function <- function(x, y){
            return(x + y)
        }

        impure_function <- function(x, y){
            print(x)
            return(x + y)
        }
    \end{lstlisting}
\end{frame}

\begin{frame}
    \centering
    \frametitle{Breakdown}
    \begin{itemize}
        \setlength{\itemsep}{2em}
        \item The pure\_function function takes two arguments, x and y, and returns their sum.
        \item The impure\_function function takes two arguments, x and y, prints the value of x, and returns their sum.
        \item The pure\_function function is a pure function because it only depends on its input values.
        \item The impure\_function function is an impure function because it has a side effect (printing the value of x).
    \end{itemize}
\end{frame}

\begin{frame}
    \centering
    \frametitle{First Class Functions}
    \begin{itemize}
        \setlength{\itemsep}{2em}
        \item In functional programming, functions are first-class citizens.
        \item This means that functions can be assigned to variables, passed as arguments, and returned from other functions.
        \item This allows for the creation of higher-order functions.
    \end{itemize}
\end{frame}

\begin{frame}[fragile]
    \begin{lstlisting}
        add <- function(x, y){
            return(x + y)
        }

        subtract <- function(x, y){
            return(x - y)
        }

        operate <- function(func, x, y){
            return(func(x, y))
        }

        operate(add, 5, 3)
        operate(subtract, 5, 3)
    \end{lstlisting}
\end{frame}

\begin{frame}
    \centering
    \frametitle{Higher Order Functions}
    \begin{itemize}
        \setlength{\itemsep}{2em}
        \item Higher-order functions are functions that can either take other functions as arguments or return them as results.
        \item This is possible because functions are first-class citizens.
        \item Higher-order functions allow us to abstract over actions, not just values.
    \end{itemize}
\end{frame}

\begin{frame}[fragile]
    \begin{lstlisting}
        add <- function(x, y){
            return(x + y)
        }

        subtract <- function(x, y){
            return(x - y)
        }

        create_operator <- function(op){
            if(op == "add"){
                return(add)
            } else if(op == "subtract"){
                return(subtract)
            }
        }

        operator <- create_operator("add")
        operator(5, 3)
    \end{lstlisting}
\end{frame}

\begin{frame}
    \centering
    \frametitle{Recursion}
    \begin{itemize}
        \setlength{\itemsep}{2em}
        \item Recursion is a technique in which a function calls itself to solve a problem.
        \item Recursion is a common feature of functional programming.
        \item Recursion is used to solve problems that can be broken down into smaller subproblems.
    \end{itemize}
\end{frame}

\begin{frame}[fragile]
    \begin{lstlisting}
        factorial <- function(n){
            if(n == 0){
                return(1)
            } else {
                return(n * factorial(n - 1))
            }
        }

        factorial(5)
    \end{lstlisting}
\end{frame}

\begin{frame}
    \centering
    \frametitle{Breakdown}
    \begin{itemize}
        \setlength{\itemsep}{2em}
        \item The factorial function takes an integer n as an argument.
        \item If n is 0, the function returns 1.
        \item Otherwise, the function returns n times the factorial of n - 1.
        \item This continues until n is 0, at which point the function returns 1.
        \item The function is called with the argument 5, which returns 120.
    \end{itemize}
\end{frame}

\begin{frame}
    \centering
    \frametitle{Tail Recursion}
    \begin{itemize}
        \setlength{\itemsep}{2em}
        \item Tail recursion is a special form of recursion where the recursive call is the last thing the function does.
        \item Tail recursion is more efficient than regular recursion because it can be optimized by the compiler.
        \item Tail recursion is a common feature of functional programming.
    \end{itemize}
\end{frame}

\begin{frame}
    \centering
    \frametitle{Map, Filter, and Reduce}
    \begin{itemize}
        \setlength{\itemsep}{2em}
        \item Map, filter, and reduce are three common higher-order functions in functional programming.
        \item Map applies a function to each element of a list.
        \item Filter selects elements from a list based on a condition.
        \item Reduce combines all elements of a list into a single value.
    \end{itemize}
\end{frame}

\begin{frame}
    \centering
    \frametitle{Map}
    \begin{itemize}
        \setlength{\itemsep}{2em}
        \item The map function applies a function to each element of a list and returns a new list with the results.
        \item The map function is a higher-order function because it takes a function as an argument.
        \item The map function is a common feature of functional programming.
    \end{itemize}
\end{frame}

\begin{frame}[fragile]
    \begin{lstlisting}
        add_one <- function(x){
            return(x + 1)
        }

        numbers <- c(1, 2, 3, 4, 5)
        mapped_numbers <- lapply(numbers, add_one)
    \end{lstlisting}
\end{frame}

\begin{frame}
    \centering
    \frametitle{Breakdown}
    \begin{itemize}
        \setlength{\itemsep}{2em}
        \item The add\_one function takes an integer x as an argument and returns x + 1.
        \item The numbers vector contains the integers 1, 2, 3, 4, and 5.
        \item The lapply function applies the add\_one function to each element of the numbers vector and returns a new list with the results.
        \item The mapped\_numbers list contains the integers 2, 3, 4, 5, and 6.
    \end{itemize}
\end{frame}

\begin{frame}
    \centering
    \frametitle{Filter}
    \begin{itemize}
        \setlength{\itemsep}{2em}
        \item The filter function selects elements from a list based on a condition and returns a new list with the selected elements.
        \item The filter function is a higher-order function because it takes a function as an argument.
        \item The filter function is a common feature of functional programming.
    \end{itemize}
\end{frame}

\begin{frame}[fragile]
    \begin{lstlisting}
        is_even <- function(x){
            return(x %% 2 == 0)
        }

        numbers <- c(1, 2, 3, 4, 5)
        filtered_numbers <- numbers[numbers %% 2 == 0]
    \end{lstlisting}
\end{frame}

\begin{frame}
    \centering
    \frametitle{Breakdown}
    \begin{itemize}
        \setlength{\itemsep}{2em}
        \item The is\_even function takes an integer x as an argument and returns TRUE if x is even and FALSE otherwise.
        \item The numbers vector contains the integers 1, 2, 3, 4, and 5.
        \item The filtered\_numbers vector contains the even integers from the numbers vector.
    \end{itemize}
\end{frame}

\begin{frame}
    \centering
    \frametitle{Reduce}
    \begin{itemize}
        \setlength{\itemsep}{2em}
        \item The reduce function combines all elements of a list into a single value using a binary operation.
        \item The reduce function is a higher-order function because it takes a function as an argument.
        \item The reduce function is a common feature of functional programming.
    \end{itemize}
\end{frame}

\begin{frame}[fragile]
    \begin{lstlisting}
        add <- function(x, y){
            return(x + y)
        }

        numbers <- c(1, 2, 3, 4, 5)
        reduced_number <- Reduce(add, numbers)
    \end{lstlisting}
\end{frame}

\begin{frame}
    \centering
    \frametitle{Breakdown}
    \begin{itemize}
        \setlength{\itemsep}{2em}
        \item The add function takes two integers, x and y, as arguments and returns their sum.
        \item The numbers vector contains the integers 1, 2, 3, 4, and 5.
        \item The reduced\_number variable contains the sum of all the integers in the numbers vector.
    \end{itemize}
\end{frame}

\begin{frame}
    \centering
    \frametitle{Functional Programming in R}
    \begin{itemize}
        \setlength{\itemsep}{2em}
        \item R is a functional programming language.
        \item R has functions and packages that make functional programming easier.
        \item R has higher-order functions like lapply, sapply, and Reduce that allow you to apply functions to lists and vectors.
    \end{itemize}
\end{frame}

\section{Data Handling}

\begin{frame}
    \frametitle{Table of Contents}
    \setbeamertemplate{section in toc}[sections numbered]
    \tableofcontents[currentsection]
\end{frame}

\begin{frame}
    \centering
    \frametitle{Data Handling}
    \begin{itemize}
        \setlength{\itemsep}{2em}
        \item Data handling is a crucial part of data analysis.
        \item R has a wide range of functions and packages that make data handling easier.
    \end{itemize}
\end{frame}

\begin{frame}
    \centering
    \frametitle{Data Structures}
    \begin{itemize}
        \setlength{\itemsep}{2em}
        \item R has several data structures that are used to store data.
        \item The most common data structures are vectors, matrices, data frames, and lists.
    \end{itemize}
\end{frame}

\begin{frame}
    \centering
    \frametitle{Lists}
    \begin{itemize}
        \setlength{\itemsep}{2em}
        \item A list is a one-dimensional array that can hold numeric, character, or logical data.
        \item Lists are created using the list() function.
        \item Lists can hold different types of data in each element.
    \end{itemize}
\end{frame}

\begin{frame}[fragile]
    \begin{lstlisting}
        list_1 <- list(1, "a", TRUE)
        list_2 <- list(
            name = "Alice",
            age = 25,
            married = TRUE
        )
    \end{lstlisting}
\end{frame}

\begin{frame}
    \centering
    \frametitle{Dictionaries}
    \begin{itemize}
        \setlength{\itemsep}{2em}
        \item A dictionary is a one-dimensional array that can hold key-value pairs.
        \item Dictionaries are created using the list() function.
        \item Dictionaries are similar to lists, but they have named elements.
    \end{itemize}
\end{frame}

\begin{frame}
    \centering
    \frametitle{Vectors}
    \begin{itemize}
        \setlength{\itemsep}{2em}
        \item A vector is a one-dimensional array that can hold numeric, character, or logical data.
        \item Vectors are created using the c() function.
        \item Vectors can be of two types: atomic vectors and lists.
    \end{itemize}
\end{frame}

\begin{frame}[fragile]
    \begin{lstlisting}
        numeric_vector <- c(1, 2, 3, 4, 5)
        character_vector <- c("a", "b", "c", "d", "e")
        logical_vector <- c(TRUE, FALSE, TRUE, FALSE, TRUE)
    \end{lstlisting}
\end{frame}

\begin{frame}
    \centering
    \frametitle{Atomic Vectors}
    \begin{itemize}
        \setlength{\itemsep}{2em}
        \item An atomic vector is a vector that can hold only one type of data.
        \item Atomic vectors can be of four types: numeric, character, logical, and complex.
        \item Atomic vectors are created using the c() function.
    \end{itemize}
\end{frame}

\begin{frame}[fragile]
    \begin{lstlisting}
        numeric_vector <- c(1, 2, 3, 4, 5)
        character_vector <- c("a", "b", "c", "d", "e")
        logical_vector <- c(TRUE, FALSE, TRUE, FALSE, TRUE)
        complex_vector <- c(1 + 2i, 3 + 4i, 5 + 6i)
    \end{lstlisting}
\end{frame}

\begin{frame}
    \centering
    \frametitle{When To Use Vectors}
    \begin{itemize}
        \setlength{\itemsep}{2em}
        \item Use vectors when you have a one-dimensional array of data.
        \item Use atomic vectors when you have a one-dimensional array of data of the same type.
        \item Use lists when you have a one-dimensional array of data of different types.
        \item Use dictionaries when you have a one-dimensional array of key-value pairs.
    \end{itemize}
\end{frame}

\begin{frame}
    \centering
    \frametitle{Matrices}
    \begin{itemize}
        \setlength{\itemsep}{2em}
        \item A matrix is a two-dimensional array that can hold numeric, character, or logical data.
        \item Matrices are created using the matrix() function.
        \item Matrices are created by combining vectors.
    \end{itemize}
\end{frame}

\begin{frame}[fragile]
    \begin{lstlisting}
        matrix_1 <- matrix(1:9, nrow = 3, ncol = 3)
        matrix_2 <- matrix(letters[1:9], nrow = 3, ncol = 3)
        matrix_3 <- matrix(
            c(TRUE, FALSE, TRUE, FALSE, TRUE, FALSE),
            nrow = 2, ncol = 3
        )
    \end{lstlisting}
\end{frame}

\begin{frame}
    \centering
    \frametitle{When To Use Matrices}
    \begin{itemize}
        \setlength{\itemsep}{2em}
        \item Use matrices when you have a two-dimensional array of data.
        \item Use matrices when you have numeric, character, or logical data.
        \item Use matrices when you want to perform matrix operations like addition, subtraction, multiplication, and division.
        \item Use matrices when you want to represent data in a tabular format.
    \end{itemize}
\end{frame}

\begin{frame}
    \centering
    \frametitle{Data Matrix Operations}
    \begin{itemize}
        \setlength{\itemsep}{2em}
        \item R has functions and packages that allow you to perform matrix operations.
        \item The most common matrix operations are addition, subtraction, multiplication, and division.
        \item R has functions that allow you to perform these operations on matrices.
        \item R has functions that allow you to transpose, invert, and concatenate matrices.
        \item R has functions that allow you to extract rows and columns from matrices.
    \end{itemize}
\end{frame}

\begin{frame}[fragile]
    \begin{lstlisting}
        matrix_1 <- matrix(1:9, nrow = 3, ncol = 3)
        matrix_2 <- matrix(9:1, nrow = 3, ncol = 3)

        # Addition
        added_matrix <- matrix_1 + matrix_2

        # Subtraction
        subtracted_matrix <- matrix_1 - matrix_2

        # Multiplication
        multiplied_matrix <- matrix_1 %*% matrix_2

        # Division
        divided_matrix <- matrix_1 / matrix_2
    \end{lstlisting}
\end{frame}

\begin{frame}
    \centering
    \frametitle{Data Frames}
    \begin{itemize}
        \setlength{\itemsep}{2em}
        \item A data frame is a two-dimensional array that can hold numeric, character, or logical data.
        \item Data frames are created using the data.frame() function.
        \item Data frames are similar to matrices, but they can hold different types of data in each column.
    \end{itemize}
\end{frame}

\begin{frame}[fragile]
    \begin{lstlisting}
        data_frame <- data.frame(
            name = c("Alice", "Bob", "Charlie"),
            age = c(25, 30, 35),
            married = c(TRUE, FALSE, TRUE)
        )
    \end{lstlisting}
\end{frame}

\begin{frame}
    \centering
    \frametitle{When To Use Data Frames}
    \begin{itemize}
        \setlength{\itemsep}{2em}
        \item Use data frames when you have a two-dimensional array of data.
        \item Use data frames when you have different types of data in each column.
        \item Use data frames when you want to perform data manipulation tasks like filtering, sorting, and aggregating data.
        \item Use data frames when you want to represent data in a tabular format.
    \end{itemize}
\end{frame}

\begin{frame}
    \centering
    \frametitle{Data Frame Operations}
    \begin{itemize}
        \setlength{\itemsep}{2em}
        \item R has functions and packages that allow you to perform data frame operations.
        \item The most common data frame operations are filtering, sorting, and aggregating data.
        \item R has functions that allow you to perform these operations on data frames.
        \item R has functions that allow you to merge and join data frames.
    \end{itemize}
\end{frame}

\begin{frame}[fragile]
    \begin{lstlisting}
        data <- data.frame(
            name = c("Alice", "Bob", "Charlie"),
            age = c(25, 30, 35),
            married = c(TRUE, FALSE, TRUE)
        )

        # Filter data
        filtered_data <- data[data$age > 30, ]

        # Sort data
        sorted_data <- data[order(data$age), ]

        # Aggregate data
        aggregated_data <- aggregate(
            data$age,
            by = list(data$married),
            FUN = mean
        )
    \end{lstlisting}
\end{frame}

\begin{frame}
    \centering
    \frametitle{File IO}
    \begin{itemize}
        \setlength{\itemsep}{2em}
        \item R has functions that allow you to read and write data from and to files.
        \item The most common file formats are CSV, Excel, and text files.
        \item R has functions that allow you to read and write data in these formats.
    \end{itemize}
\end{frame}

\begin{frame}[fragile]
    \begin{lstlisting}
        data <- read.csv("data_frame.csv")
        write.csv(data, "data_frame.csv")
    \end{lstlisting}
\end{frame}

\begin{frame}
    \centering
    \frametitle{Word of Caution}
    \begin{itemize}
        \setlength{\itemsep}{2em}
        \item If you specify the same file name for the read and write functions, the file will be overwritten.
        \item Make sure to back up your data before using the write function.
        \item Make sure to check the file permissions before using the write function.
    \end{itemize}
\end{frame}

\begin{frame}
    \centering
    \frametitle{Data Manipulation}
    \begin{itemize}
        \setlength{\itemsep}{2em}
        \item Data manipulation is the process of transforming data to make it more useful for analysis.
        \item R has functions and packages that make data manipulation easier.
        \item The most common data manipulation tasks are filtering, sorting, and aggregating data.
    \end{itemize}
\end{frame}

\begin{frame}[fragile]
    \begin{lstlisting}
        data <- data.frame(
            name = c("Alice", "Bob", "Charlie"),
            age = c(25, 30, 35),
            married = c(TRUE, FALSE, TRUE)
        )

        # Filter data
        filtered_data <- data[data$age > 30, ]

        # Sort data
        sorted_data <- data[order(data$age), ]

        # Aggregate data
        aggregated_data <- aggregate(
            data$age,
            by = list(data$married),
            FUN = mean
        )
    \end{lstlisting}
\end{frame}

\begin{frame}
    \centering
    \frametitle{Merge and Join}
    \begin{itemize}
        \setlength{\itemsep}{2em}
        \item Merge and join are two common data manipulation tasks.
        \item Merge is used to combine two data frames based on a common column.
        \item Join is used to combine two data frames based on a common column.
    \end{itemize}
\end{frame}

\begin{frame}[fragile]
    \begin{lstlisting}
        data_1 <- data.frame(
            name = c("Alice", "Bob", "Charlie"),
            age = c(25, 30, 35),
            married = c(TRUE, FALSE, TRUE)
        )

        data_2 <- data.frame(
            name = c("Alice", "Bob", "Charlie"),
            salary = c(50000, 60000, 70000)
        )

        merged_data <- merge(data_1, data_2, by = "name")
        joined_data <- merge(
            data_1, data_2, by = "name",
            all = TRUE
        )
    \end{lstlisting}
\end{frame}

\begin{frame}
    \centering
    \frametitle{Breakdown}
    \begin{itemize}
        \setlength{\itemsep}{2em}
        \item The data\_1 data frame contains the name, age, and married columns.
        \item The data\_2 data frame contains the name and salary columns.
        \item The merged\_data data frame contains the name, age, married, and salary columns.
        \item The joined\_data data frame contains the name, age, married, and salary columns.
    \end{itemize}
\end{frame}

\begin{frame}
    \centering
    \frametitle{Tidyverse}
    \begin{itemize}
        \setlength{\itemsep}{2em}
        \item Tidyverse is a collection of R packages that make data manipulation easier.
        \item Tidyverse packages are designed to work together and follow a consistent design philosophy.
        \item Tidyverse packages are widely used in the R community.
    \end{itemize}
\end{frame}

\begin{frame}
    \centering
    \frametitle{Breakdown}
    \begin{itemize}
        \setlength{\itemsep}{2em}
        \item The dplyr package provides functions for data manipulation tasks like filtering, sorting, and aggregating data.
        \item The tidyr package provides functions for data manipulation tasks like reshaping and tidying data.
        \item The ggplot2 package provides functions for data visualization tasks like creating plots and charts.
        \item The readr package provides functions for reading and writing data from and to files.
    \end{itemize}
\end{frame}

\begin{frame}
    \centering
    \frametitle{dplyr}
    \begin{itemize}
        \setlength{\itemsep}{2em}
        \item The dplyr package provides functions for data manipulation tasks like filtering, sorting, and aggregating data.
        \item Some common dplyr functions are filter(), arrange(), and summarise().
        \item filter() is used to filter rows based on a condition.
        \item arrange() is used to sort rows based on a column.
        \item summarise() is used to aggregate data based on a column.
    \end{itemize}
\end{frame}

\begin{frame}[fragile]
    \begin{lstlisting}
        library(dplyr)

        data <- data.frame(
            name = c("Alice", "Bob", "Charlie"),
            age = c(25, 30, 35),
            married = c(TRUE, FALSE, TRUE)
        )

        filtered_data <- data %>% filter(age > 30)
        sorted_data <- data %>% arrange(age)
        aggregated_data <- data %>% summarise(mean_age = mean(age))
    \end{lstlisting}
\end{frame}

\begin{frame}
    \centering
    \frametitle{Breakdown}
    \begin{itemize}
        \setlength{\itemsep}{2em}
        \item The data data frame contains the name, age, and married columns.
        \item The filtered\_data data frame contains the rows where the age column is greater than 30.
        \item The sorted\_data data frame contains the rows sorted by the age column.
        \item The aggregated\_data data frame contains the mean age of the data frame.
    \end{itemize}
\end{frame}

\begin{frame}
    \centering
    \frametitle{tidyr}
    \begin{itemize}
        \setlength{\itemsep}{2em}
        \item The tidyr package provides functions for data manipulation tasks like reshaping and tidying data.
        \item Some common tidyr functions are gather() and spread().
        \item gather() is used to reshape data from wide to long format.
        \item spread() is used to reshape data from long to wide format.
    \end{itemize}
\end{frame}

\begin{frame}[fragile]
    \begin{lstlisting}
        library(tidyr)

        data <- data.frame(
            name = c("Alice", "Bob", "Charlie"),
            age = c(25, 30, 35),
            married = c(TRUE, FALSE, TRUE)
        )

        gathered_data <- data %>% gather(key = "variable", value = "value", -name)
        spreaded_data <- gathered_data %>% spread(key = "variable", value = "value")
    \end{lstlisting}
\end{frame}

\begin{frame}
    \centering
    \frametitle{Breakdown}
    \begin{itemize}
        \setlength{\itemsep}{2em}
        \item The data data frame contains the name, age, and married columns.
        \item The gathered\_data data frame contains the data in long format.
        \item The spreaded\_data data frame contains the data in wide format.
    \end{itemize}
\end{frame}

\begin{frame}
    \centering
    \frametitle{ggplot2}
    \begin{itemize}
        \setlength{\itemsep}{2em}
        \item The ggplot2 package provides functions for data visualization tasks like creating plots and charts.
        \item The ggplot2 package is based on the grammar of graphics.
        \item The ggplot2 package allows you to create complex plots with a few lines of code.
    \end{itemize}
\end{frame}

\begin{frame}[fragile]
    \begin{lstlisting}
        library(ggplot2)

        data <- data.frame(
            name = c("Alice", "Bob", "Charlie"),
            age = c(25, 30, 35),
            married = c(TRUE, FALSE, TRUE)
        )

        ggplot(data, aes(x = name, y = age)) +
            geom_bar(stat = "identity")
    \end{lstlisting}
\end{frame}

\begin{frame}
    \centering
    \frametitle{Breakdown}
    \begin{itemize}
        \setlength{\itemsep}{2em}
        \item The data data frame contains the name, age, and married columns.
        \item The ggplot function creates a plot with the data data frame.
        \item The aes function specifies the x and y variables for the plot.
        \item The geom\_bar function creates a bar plot with the data.
    \end{itemize}
\end{frame}

\begin{frame}
    \centering
    \frametitle{readr}
    \begin{itemize}
        \setlength{\itemsep}{2em}
        \item The readr package provides functions for reading and writing data from and to files.
        \item The readr package is faster and more user-friendly than the base R functions.
        \item The readr package is widely used in the R community.
    \end{itemize}
\end{frame}

\begin{frame}[fragile]
    \begin{lstlisting}
        library(readr)

        data <- read_csv("data_frame.csv")
        write_csv(data, "data_frame.csv")
    \end{lstlisting}
\end{frame}

\begin{frame}
    \centering
    \frametitle{Breakdown}
    \begin{itemize}
        \setlength{\itemsep}{2em}
        \item The read\_csv function reads data from a CSV file.
        \item The write\_csv function writes data to a CSV file.
        \item The data data frame contains the data read from the CSV file.
    \end{itemize}
\end{frame}

\begin{frame}
    \centering
    \frametitle{Summary}
    \begin{itemize}
        \setlength{\itemsep}{2em}
        \item Functional programming is a programming paradigm that treats computation as the evaluation of mathematical functions.
        \item Functional programming focuses on immutability, pure functions, and first-class functions.
        \item R is a functional programming language that has functions and packages that make functional programming easier.
        \item R has data structures like vectors, matrices, data frames, and lists that are used to store data.
        \item R has functions and packages that allow you to perform data manipulation tasks like filtering, sorting, and aggregating data.
    \end{itemize}
\end{frame}

\end{document}