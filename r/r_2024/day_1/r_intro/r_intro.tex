\documentclass[serif, 9pt, aspectratio=32]{beamer} 
\usetheme{Darmstadt}

\usepackage{appendixnumberbeamer}
\usepackage{booktabs}
\usepackage[scale=2]{ccicons}
\usepackage{pgfplots}
\usepgfplotslibrary{dateplot}
\usepackage{xspace}
\usepackage{tikz}
\usepackage{hyperref}
\usepackage{xcolor}
\usepackage{listings}
\usetikzlibrary{shapes, arrows}
\newcommand{\themename}{\textbf{\textsc{metropolis}}\xspace}

\title{Introduction to R}
\date{\today}
\author{Tan Sein Jone}
\institute{University of British Columbia}

\pgfplotsset{compat=1.18}
\setbeamertemplate{footline}[frame number]

\begin{document}

\maketitle

\begin{frame}{Table of contents}
    \setbeamertemplate{section in toc}[sections numbered]
    \tableofcontents[hideallsubsections]
\end{frame}

\section{Functional Programming}

\begin{frame}
    \frametitle{Table of Contents}
    \setbeamertemplate{section in toc}[sections numbered]
    \tableofcontents[currentsection]
\end{frame}

\begin{frame}
    \frametitle{Functional Programming}
    \begin{itemize}
        \item Functional programming is a programming paradigm that treats computation as the evaluation of mathematical functions and avoids changing-state and mutable data.
        \item It is a declarative type of programming style.
        \item It focuses on what to solve rather than how to solve.
        \item It uses expressions instead of statements.
        \item It is based on mathematical functions.
    \end{itemize}
\end{frame}

\begin{frame}
    \centering
    \frametitle{Pure Functions}
    \begin{itemize}
        \setlength{\itemsep}{2em}
        \item A pure function is a function where the output value is determined by its input values, without observable side effects.
        \item This is how functions in math work: Math.cos(x) will, for the same value of x, always return the same result.
        \item Pure functions are easier to reason about and test.
    \end{itemize}
\end{frame}

\begin{frame}[fragile]
    \begin{lstlisting}
        pure_function <- function(x, y){
            return(x + y)
        }

        impure_function <- function(x, y){
            print(x)
            return(x + y)
        }
    \end{lstlisting}
\end{frame}

\begin{frame}
    \centering
    \frametitle{First Class Functions}
    \begin{itemize}
        \setlength{\itemsep}{2em}
        \item In functional programming, functions are first-class citizens.
        \item This means that functions can be assigned to variables, passed as arguments, and returned from other functions.
        \item This allows for the creation of higher-order functions.
    \end{itemize}
\end{frame}

\begin{frame}[fragile]
    \begin{lstlisting}
        add <- function(x, y){
            return(x + y)
        }

        subtract <- function(x, y){
            return(x - y)
        }

        operate <- function(func, x, y){
            return(func(x, y))
        }

        operate(add, 5, 3)
        operate(subtract, 5, 3)
    \end{lstlisting}
\end{frame}

\begin{frame}
    \centering
    \frametitle{Higher Order Functions}
    \begin{itemize}
        \setlength{\itemsep}{2em}
        \item Higher-order functions are functions that can either take other functions as arguments or return them as results.
        \item This is possible because functions are first-class citizens.
        \item Higher-order functions allow us to abstract over actions, not just values.
    \end{itemize}
\end{frame}

\begin{frame}[fragile]
    \begin{lstlisting}
        add <- function(x, y){
            return(x + y)
        }

        subtract <- function(x, y){
            return(x - y)
        }

        create_operator <- function(op){
            if(op == "add"){
                return(add)
            } else if(op == "subtract"){
                return(subtract)
            }
        }

        operator <- create_operator("add")
        operator(5, 3)
    \end{lstlisting}
\end{frame}

\section{Data Handling}

\begin{frame}
    \frametitle{Table of Contents}
    \setbeamertemplate{section in toc}[sections numbered]
    \tableofcontents[currentsection]
\end{frame}

\begin{frame}
    \centering
    \frametitle{Data Handling}
    \begin{itemize}
        \setlength{\itemsep}{2em}
        \item Data handling is a crucial part of data analysis.
        \item R has a wide range of functions and packages that make data handling easier.
    \end{itemize}
\end{frame}

\begin{frame}
    \centering
    \frametitle{Data Structures}
    \begin{itemize}
        \setlength{\itemsep}{2em}
        \item R has several data structures that are used to store data.
        \item The most common data structures are vectors, matrices, data frames, and lists.
    \end{itemize}
\end{frame}

\begin{frame}
    \centering
    \frametitle{Vectors}
    \begin{itemize}
        \setlength{\itemsep}{2em}
        \item A vector is a one-dimensional array that can hold numeric, character, or logical data.
        \item Vectors are created using the c() function.
        \item Vectors can be of two types: atomic vectors and lists.
    \end{itemize}
\end{frame}

\begin{frame}[fragile]
    \begin{lstlisting}
        numeric_vector <- c(1, 2, 3, 4, 5)
        character_vector <- c("a", "b", "c", "d", "e")
        logical_vector <- c(TRUE, FALSE, TRUE, FALSE, TRUE)
    \end{lstlisting}
\end{frame}

\begin{frame}
    \centering
    \frametitle{Matrices}
    \begin{itemize}
        \setlength{\itemsep}{2em}
        \item A matrix is a two-dimensional array that can hold numeric, character, or logical data.
        \item Matrices are created using the matrix() function.
        \item Matrices are created by combining vectors.
    \end{itemize}
\end{frame}

\begin{frame}[fragile]
    \begin{lstlisting}
        matrix_1 <- matrix(1:9, nrow = 3, ncol = 3)
        matrix_2 <- matrix(letters[1:9], nrow = 3, ncol = 3)
        matrix_3 <- matrix(c(TRUE, FALSE, TRUE, FALSE, TRUE, FALSE), nrow = 2, ncol = 3)
    \end{lstlisting}
\end{frame}

\begin{frame}
    \centering
    \frametitle{Data Frames}
    \begin{itemize}
        \setlength{\itemsep}{2em}
        \item A data frame is a two-dimensional array that can hold numeric, character, or logical data.
        \item Data frames are created using the data.frame() function.
        \item Data frames are similar to matrices, but they can hold different types of data in each column.
    \end{itemize}
\end{frame}

\begin{frame}[fragile]
    \begin{lstlisting}
        data_frame <- data.frame(
            name = c("Alice", "Bob", "Charlie"),
            age = c(25, 30, 35),
            married = c(TRUE, FALSE, TRUE)
        )
    \end{lstlisting}
\end{frame}

\begin{frame}
    \centering
    \frametitle{File IO}
    \begin{itemize}
        \setlength{\itemsep}{2em}
        \item R has functions that allow you to read and write data from and to files.
        \item The most common file formats are CSV, Excel, and text files.
        \item R has functions that allow you to read and write data in these formats.
    \end{itemize}
\end{frame}

\begin{frame}[fragile]
    \begin{lstlisting}
        data <- read.csv("data_frame.csv")
        write.csv(data, "data_frame.csv")
    \end{lstlisting}
\end{frame}

\begin{frame}
    \centering
    \frametitle{Data Manipulation}
    \begin{itemize}
        \setlength{\itemsep}{2em}
        \item Data manipulation is the process of transforming data to make it more useful for analysis.
        \item R has functions and packages that make data manipulation easier.
        \item The most common data manipulation tasks are filtering, sorting, and aggregating data.
    \end{itemize}
\end{frame}

\begin{frame}[fragile]
    \begin{lstlisting}
        data <- data.frame(
            name = c("Alice", "Bob", "Charlie"),
            age = c(25, 30, 35),
            married = c(TRUE, FALSE, TRUE)
        )

        # Filter data
        filtered_data <- data[data$age > 30, ]

        # Sort data
        sorted_data <- data[order(data$age), ]

        # Aggregate data
        aggregated_data <- aggregate(data$age, by = list(data$married), FUN = mean)
    \end{lstlisting}
\end{frame}

\begin{frame}
    \centering
    \frametitle{Merge and Join}
    \begin{itemize}
        \setlength{\itemsep}{2em}
        \item Merge and join are two common data manipulation tasks.
        \item Merge is used to combine two data frames based on a common column.
        \item Join is used to combine two data frames based on a common column.
    \end{itemize}
\end{frame}

\begin{frame}[fragile]
    \begin{lstlisting}
        data_1 <- data.frame(
            name = c("Alice", "Bob", "Charlie"),
            age = c(25, 30, 35),
            married = c(TRUE, FALSE, TRUE)
        )

        data_2 <- data.frame(
            name = c("Alice", "Bob", "Charlie"),
            salary = c(50000, 60000, 70000)
        )

        merged_data <- merge(data_1, data_2, by = "name")
        joined_data <- merge(data_1, data_2, by = "name", all = TRUE)
    \end{lstlisting}
\end{frame}

\end{document}