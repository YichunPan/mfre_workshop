\documentclass[serif, 9pt, aspectratio=32]{beamer} 
\usetheme{Darmstadt}

\usepackage{appendixnumberbeamer}
\usepackage{booktabs}
\usepackage[scale=2]{ccicons}
\usepackage{pgfplots}
\usepgfplotslibrary{dateplot}
\usepackage{xspace}
\usepackage{tikz}
\usepackage{hyperref}
\usepackage{xcolor}
\usepackage{listings}
\usetikzlibrary{shapes, arrows}
\newcommand{\themename}{\textbf{\textsc{metropolis}}\xspace}

\title{R Applications}
\date{\today}
\author{Tan Sein Jone}
\institute{University of British Columbia}

\pgfplotsset{compat=1.18}
\setbeamertemplate{footline}[frame number]

\begin{document}

\maketitle

\begin{frame}{Table of contents}
    \setbeamertemplate{section in toc}[sections numbered]
    \tableofcontents[hideallsubsections]
\end{frame}

\section{Hypothesis Testing}

\begin{frame}
    \frametitle{Table of Contents}
    \setbeamertemplate{section in toc}[sections numbered]
    \tableofcontents[currentsection]
\end{frame}

\begin{frame}{Hypothesis Testing}
    \begin{itemize}
        \item Hypothesis testing is a statistical method that is used in making statistical decisions using experimental data.
        \item A hypothesis test evaluates two mutually exclusive statements about a population to determine which statement is best supported by the sample data.
        \item These two statements are called the null hypothesis and the alternative hypothesis.
    \end{itemize}
\end{frame}

\begin{frame}{Hypothesis Testing}
    \begin{itemize}
        \setlength{\itemsep}{2em}
        \item The null hypothesis is the statement being tested. Usually, the null hypothesis states that there is no effect or no difference.
        \item The alternative hypothesis is the statement that is accepted if the sample data provide enough evidence that the null hypothesis is false.
        \item The alternative hypothesis states that there is an effect or a difference.
    \end{itemize}
\end{frame}

\begin{frame}{Hypothesis Testing}
    \begin{itemize}
        \setlength{\itemsep}{2em}
        \item The hypothesis test is conducted by comparing the value of the test statistic to a critical value.
        \item The critical value is a value that determines whether the null hypothesis can be rejected.
        \item If the test statistic is more extreme than the critical value, then the null hypothesis is rejected.
    \end{itemize}
\end{frame}

\begin{frame}{Hypothesis Testing}
    \begin{itemize}
        \setlength{\itemsep}{2em}
        \item The p-value is the probability of observing a test statistic as extreme as the one computed from the sample data, assuming that the null hypothesis is true.
        \item If the p-value is less than the significance level, then the null hypothesis is rejected.
        \item The significance level is the probability of rejecting the null hypothesis when it is true.
    \end{itemize}
\end{frame}

\begin{frame}{Hypothesis Testing}
    \begin{itemize}
        \setlength{\itemsep}{2em}
        \item The p-value is a measure of the strength of the evidence against the null hypothesis.
        \item The smaller the p-value, the stronger the evidence against the null hypothesis.
        \item The p-value is compared to the significance level to determine whether the null hypothesis should be rejected.
    \end{itemize}
\end{frame}

\begin{frame}{Hypothesis Testing in R}
    \begin{itemize}
        \setlength{\itemsep}{2em}
        \item In R, the \texttt{t.test()} function is used to perform hypothesis tests.
        \item The \texttt{t.test()} function takes in the sample data and the null hypothesis as arguments.
        \item The function returns the test statistic, the p-value, and the confidence interval.
    \end{itemize}
\end{frame}

\begin{frame}{Hypothesis Testing in R}
    \begin{itemize}
        \setlength{\itemsep}{2em}
        \item The \texttt{t.test()} function can be used to perform one-sample t-tests, two-sample t-tests, and paired t-tests.
        \item The \texttt{t.test()} function can also be used to perform one-sample z-tests and two-sample z-tests.
        \item The \texttt{t.test()} function can be used to perform hypothesis tests for means, proportions, and variances.
    \end{itemize}
\end{frame}

\begin{frame}[fragile]{Hypothesis Testing in R}
    \begin{lstlisting}[language=R]
# Generate some data
set.seed(123)
x <- rnorm(100, mean = 5, sd = 2)
y <- rnorm(100, mean = 6, sd = 2)

# One-sample t-test
t.test(x, mu = 0)

# Two-sample t-test
t.test(x, y)

# Paired t-test
t.test(x, y, paired = TRUE)
    \end{lstlisting}
\end{frame}

\section{Regression Analysis}

\begin{frame}
    \frametitle{Table of Contents}
    \setbeamertemplate{section in toc}[sections numbered]
    \tableofcontents[currentsection]
\end{frame}

\begin{frame}{Regression Analysis}
    \begin{itemize}
        \item Regression analysis is a statistical method that is used to model the relationship between a dependent variable and one or more independent variables.
        \item The goal of regression analysis is to estimate the parameters of the regression model that best fit the data.
        \item The regression model is a mathematical equation that describes the relationship between the dependent variable and the independent variables.
    \end{itemize}
\end{frame}

\begin{frame}{Regression Analysis}
    \begin{itemize}
        \setlength{\itemsep}{2em}
        \item There are many types of regression models, such as linear regression, logistic regression, and polynomial regression.
        \item Linear regression is a regression model that assumes a linear relationship between the dependent variable and the independent variables.
        \item Logistic regression is a regression model that is used when the dependent variable is binary.
    \end{itemize}
\end{frame}

\begin{frame}{Regression Analysis}
    \begin{itemize}
        \setlength{\itemsep}{2em}
        \item Polynomial regression is a regression model that is used when the relationship between the dependent variable and the independent variables is not linear.
        \item The regression model is estimated using the method of least squares, which minimizes the sum of the squared differences between the observed values and the predicted values.
        \item The estimated parameters of the regression model are the coefficients of the independent variables.
    \end{itemize}
\end{frame}

\begin{frame}{Regression Analysis}
    \begin{itemize}
        \setlength{\itemsep}{2em}
        \item The goodness of fit of the regression model is measured using the coefficient of determination, which is the proportion of the variance in the dependent variable that is explained by the independent variables.
        \item The coefficient of determination ranges from 0 to 1, with higher values indicating a better fit.
        \item The significance of the regression model is tested using the F-test, which tests whether the regression model is a better fit than a model with no independent variables.
    \end{itemize}
\end{frame}

\begin{frame}{Regression Analysis in R}
    \begin{itemize}
        \setlength{\itemsep}{2em}
        \item In R, the \texttt{lm()} function is used to fit linear regression models.
        \item The \texttt{lm()} function takes in the formula for the regression model and the data as arguments.
        \item The formula specifies the dependent variable and the independent variables in the regression model.
    \end{itemize}
\end{frame}

\begin{frame}{Regression Analysis in R}
    \begin{itemize}
        \setlength{\itemsep}{2em}
        \item The \texttt{summary()} function is used to display the results of the regression analysis.
        \item The \texttt{summary()} function displays the estimated coefficients, the standard errors, the t-values, and the p-values of the regression model.
        \item The \texttt{summary()} function also displays the coefficient of determination and the results of the F-test.
    \end{itemize}
\end{frame}

\begin{frame}[fragile]{Regression Analysis in R}
    \begin{lstlisting}[language=R]
# Generate some data
set.seed(123)
data <- data.frame(
    y = rnorm(100, mean = 5, sd = 2),
    x1 = rnorm(100, mean = 6, sd = 2),
    x2 = rnorm(100, mean = 7, sd = 2)
)

# Fit linear regression model
model <- lm(y ~ x1 + x2, data = data)

# Display results
summary(model)
    \end{lstlisting}
\end{frame}

% \section{Time Series Analysis}

% \begin{frame}
%     \frametitle{Table of Contents}
%     \setbeamertemplate{section in toc}[sections numbered]
%     \tableofcontents[currentsection]
% \end{frame}

% \begin{frame}{Time Series Analysis}
%     \begin{itemize}
%         \item Time series analysis is a statistical method that is used to model and forecast time series data.
%         \item Time series data is a sequence of observations that are recorded at regular intervals over time.
%         \item Time series analysis is used in many fields, such as economics, finance, and engineering, to analyze and predict trends and patterns in the data.
%     \end{itemize}
% \end{frame}

% \begin{frame}{Time Series Analysis}
%     \begin{itemize}
%         \setlength{\itemsep}{2em}
%         \item Time series analysis is used to model the underlying structure of the time series data and to make forecasts of future values.
%         \item The goal of time series analysis is to identify the patterns and trends in the data and to make predictions based on these patterns and trends.
%         \item Time series analysis is used to analyze and forecast time series data using statistical models, such as autoregressive integrated moving average (ARIMA) models and exponential smoothing models.
%     \end{itemize}
% \end{frame}

% \begin{frame}{Time Series Analysis}
%     \begin{itemize}
%         \setlength{\itemsep}{2em}
%         \item ARIMA models are a class of models that are used to model time series data that exhibit autocorrelation and seasonality.
%         \item ARIMA models are specified by three parameters: the autoregressive order (p), the differencing order (d), and the moving average order (q).
%         \item Exponential smoothing models are a class of models that are used to model time series data that exhibit trend and seasonality.
%     \end{itemize}
% \end{frame}

% \begin{frame}{Time Series Analysis}
%     \begin{itemize}
%         \setlength{\itemsep}{2em}
%         \item Exponential smoothing models are specified by three parameters: the level parameter (alpha), the trend parameter (beta), and the seasonality parameter (gamma).
%         \item The parameters of the time series models are estimated using the method of maximum likelihood, which maximizes the likelihood of the observed data given the model.
%         \item The goodness of fit of the time series model is measured using the mean squared error (MSE) and the Akaike information criterion (AIC).
%     \end{itemize}
% \end{frame}

% \begin{frame}{Time Series Analysis in R}
%     \begin{itemize}
%         \setlength{\itemsep}{2em}
%         \item In R, the \texttt{arima()} function is used to fit ARIMA models to time series data.
%         \item The \texttt{arima()} function takes in the time series data and the parameters of the ARIMA model as arguments.
%         \item The parameters of the ARIMA model are specified by the order argument, which is a vector of the autoregressive order, the differencing order, and the moving average order.
%     \end{itemize}
% \end{frame}

% \begin{frame}{Time Series Analysis in R}
%     \begin{itemize}
%         \setlength{\itemsep}{2em}
%         \item In R, the \texttt{ets()} function is used to fit exponential smoothing models to time series data.
%         \item The \texttt{ets()} function takes in the time series data and the parameters of the exponential smoothing model as arguments.
%         \item The parameters of the exponential smoothing model are specified by the model argument, which is a string that specifies the type of exponential smoothing model to fit.
%     \end{itemize}
% \end{frame}

% \begin{frame}[fragile]{Time Series Analysis in R}
%     \begin{lstlisting}[language=R]
% # Load packages
% library(forecast)

% # Generate time series data
% data <- ts(data, start = 1, end = n, frequency = f)

% # Fit ARIMA model
% model <- arima(data, order = c(p, d, q))

% # Fit exponential smoothing model
% model <- ets(data, model = "ZZZ")
%     \end{lstlisting}
% \end{frame}

\section{Research Design}

\begin{frame}
    \frametitle{Table of Contents}
    \setbeamertemplate{section in toc}[sections numbered]
    \tableofcontents[currentsection]
\end{frame}

\begin{frame}{Instrumental Variables}
    \begin{itemize}
        \setlength{\itemsep}{2em}
        \item Instrumental variables are used in econometrics to estimate the causal effect of an independent variable on a dependent variable.
        \item Instrumental variables are used when the independent variable is correlated with the error term in the regression model.
        \item Instrumental variables are used to identify the causal effect of the independent variable by removing the correlation between the independent variable and the error term.
    \end{itemize}
\end{frame}

\begin{frame}{Instrumental Variables}
    \begin{itemize}
        \setlength{\itemsep}{2em}
        \item Instrumental variables are variables that are correlated with the independent variable but are uncorrelated with the error term.
        \item Instrumental variables are used to estimate the causal effect of the independent variable by using the variation in the instrumental variables to identify the causal effect.
        \item Instrumental variables are used in regression analysis to estimate the parameters of the regression model that best fit the data.
    \end{itemize}
\end{frame}

\begin{frame}{Instrumental Variables in R}
    \begin{itemize}
        \setlength{\itemsep}{2em}
        \item In R, the \texttt{ivreg()} function is used to estimate instrumental variables regression models.
        \item The \texttt{ivreg()} function takes in the formula for the regression model, the data, and the instrumental variables as arguments.
        \item The formula specifies the dependent variable, the independent variables, and the instrumental variables in the regression model.
    \end{itemize}
\end{frame}

\begin{frame}{Instrumental Variables in R}
    \begin{itemize}
        \setlength{\itemsep}{2em}
        \item The \texttt{ivreg()} function estimates the parameters of the regression model using the method of instrumental variables.
        \item The \texttt{ivreg()} function returns the estimated coefficients, the standard errors, the t-values, and the p-values of the regression model.
        \item The \texttt{ivreg()} function is used to estimate the causal effect of the independent variable on the dependent variable by removing the correlation between the independent variable and the error term.
    \end{itemize}
\end{frame}

\begin{frame}[fragile]{Instrumental Variables in R}
    \begin{lstlisting}[language=R]
library(AER)
# Generate some data 
set.seed(123)
n <- 100
x <- rnorm(n)
z <- rnorm(n)
y <- 1 + 2 * x + 3 * z + rnorm(n)

# Fit instrumental variables regression model
model <- ivreg(y ~ x | z)
summary(model)
    \end{lstlisting}
\end{frame}

\end{document}